\chapter{Códigos utilizados}

Exemplo de arquivo de configuração utilizado para ativar o INTSpect. Nele definimos:

\begin{itemize}
    \item o mecanismo a ser testado;
    \item a quantidade de informações que podem ser registradas por medição;
    \item o tempo entre as medições;
    \item o tipo de mecanismo para contar o tempo entre as medições;
    \item o intervalo de medições para ter um retorno no console;
    \item a quantidade de medições a serem realizadas por teste;
\end{itemize}

\begin{listing}[!htb]
\inputminted[linenos,tabsize=2,breaklines]{ini}{codigos/softirq}
\caption{Exemplo de arquivo de configuração do INTSight}
\label{codigo:config}
\end{listing}

\newpage
Código Python unir os 2 arquivos CSV gerados pelo teste em um único. O código percorre as duas matrizes e gera uma nova matriz onde cada medição está em uma linha.

\begin{listing}[!htb]
\inputminted[linenos,tabsize=2,breaklines]{python}{codigos/tidy-chunk-result.py}
\caption{Código para tratar os dados}
\label{codigo:consolidate}
\end{listing}

\newpage
Código Python com Spark para consolidar os dados dos testes e extrair algumas estatísticas.

\begin{listing}[!htb]
\inputminted[linenos,tabsize=2,breaklines]{python}{codigos/summary-pyspark.py}
\caption{Código para consolidar os dados}
\label{codigo:summary}
\end{listing}

\newpage
Código do programa em C para gerar sobrecarga no processador.

\begin{listing}[!htb]
\inputminted[linenos,tabsize=2,breaklines]{c}{codigos/cpuload.c}
\caption{Código para sobrecarregar a CPU}
\label{codigo:sobrecarga}
\end{listing}
