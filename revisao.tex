\xchapter{Revisão bibliográfica}{Base teórica}

O que outras pessoas fizeram sobre esse tema.

\section{Sistemas Operacionais de tempo real}

Entre os desenvolvedores existe um ditado que é "um sistema de tempo real é um sistema que faz o que você espera que ele faça no tempo que você espera que ele faça". Assim, qualquer sistema pode ser considerado um sistema de tempo real dado a restrição para realizar a sua tarefa. Em alguns casos o cumprimento da tarefa fora do prazo esperado é apenas desagradável, porém em outros casos esse atraso pode comprometer todo o funcionamento do sistema. 

Em sistema de controle ou segurança, por exemplo, um atraso em determinada ação, como disparar um alarme ou ativar um controle anti-incêndio, pode levar a vítimas fatais. Laplante \cite{Laplante2004} define um sistema de tempo real como "um sistema de tempo real é aquele que deve satisfazer explicitamente restrições de tempo de resposta podendo ter consequências de risco ou falha não satisfazendo às suas restrições."

\subsection{Preempção}
\section{Linux}
\subsection{Interrupções}
\subsubsection{Tipos de Interrupções}
\subsection{Latências de interrupção}

\section{Quem fez o quê}

Tanembaum \cite{Tanenbaum2016}

Paul, testes \cite{Regnier2008}

Interrupt response times on Arduino and Raspberry Pi \cite{Solc2016}

INTSight \cite{Gerhorst2018}

Mais...

\section{Como fizeram}
\section{Mais algo?}