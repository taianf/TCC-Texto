\resumo
A popularização dos VANTs - Veículos Aéreos Não Tripulados, mais conhecidos como drones, trouxe praticidade e comodidade para executar várias tarefas que sem sua ajuda seriam muito trabalhosas ou mesmo muito perigosas. Segunda a ANAC, Agência Nacional de Aviação Civil, em julho de 2019 o Brasil já tinha mais de 70 mil drones cadastrados, sendo mais de 25 mil para uso profissional e mais de 45 mil para uso recreativo. Para algumas atividades mais simples o controle manual de um drone é suficiente mas para atividades mais extensivas ou repetitivas o controle manual não é apropriado. Alguns drones mais simples não possuem capacidade de adicionar mais sensores e uma alternativa comum é acoplar um sistema embarcado, como um Raspberry Pi, para controlar o drone através da API do fabricante do drone. Este sistema embarcado deve sempre ser capaz de responder dentro de um período crítico, pois um atraso na leitura do sensor ou no controle do drone, dependendo de sua aplicação, pode levar a falhas catastróficas e potencialmente colocar em risco a vida humana. Neste contexto, este trabalho tem como objetivo avaliar o tempo de resposta de Raspberry Pi 3 Modelo B, modelo utilizado no LaSiD - Laboratório de Sistemas Distribuídos para pesquisas com o drone AR.Drone 2.0 fabricado pela Parrot, em diferentes cenários e em diferentes implementações do kernel do linux, como o patch PREEMPT-RT, para ajudar os projetistas de aplicativos a desenvolver sistemas de controle de drones.

% Palavras-chave do resumo em Portugues
\begin{keywords}
Raspberry Pi; Sistemas de tempo real; PREEMPT-RT; Drones; Sistemas embarcados
\end{keywords}
