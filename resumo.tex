\resumo

Os sistemas modernos de computadores são capazes de executar muitas tarefas e também precisam atender a uma variedade de requisitos. Alguns sistemas têm requisitos temporais e precisam responder rápida e corretamente a eventos externos, como carros, aeronaves, sistemas robóticos e outros sistemas mecatrônicos, nos quais uma resposta atrasada pode comprometer a integridade do sistema. Para isso, sistemas operacionais específicos foram criados para atender a essa classe de requisitos, os sistemas operacionais de tempo real. Esses eventos externos são sinalizados para o processador através do mecanismo de interrupção. Para manter o sistema responsivo, essas interrupções precisam ser tratadas em um tempo curto e previsível. Como os sistemas são cada vez mais complexos, modelos de análise precisos não existem, portanto as estratégias de medição são a melhor maneira de verificar a responsividade do sistema. Um dos sistemas compactos mais populares é o Raspberry Pi, uma plataforma pequena, de baixo consumo energético e baixo custo que usa o Linux, um dos sistemas operacionais mais populares do mundo. Isso faz do Raspberry Pi uma das plataformas mais amplamente usadas para o desenvolvimento de sistemas embarcados em vários projetos. Este trabalho analisa a latência do Raspberry ao responder a eventos externos no Linux padrão e no Preempt-RT, um patch em tempo real para Linux.

% Palavras-chave do resumo em Portugues
\begin{keywords}
Raspberry Pi; Sistemas de tempo real; Preempt-RT; Sistemas embarcados
\end{keywords}
