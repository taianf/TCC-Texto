\xchapter{Introdução}{Opcional}

 % A história dos drones pode parecer recente diante dos avanços tecnológicos que baratearam as tecnologias envolvidas nos modelos mais conhecidos, mas, ao buscar sua origem, percebemos que estamos usando drones há mais de um século. No livro \textit{The Future of Drone Use: Opportunities and Threats from Ethical and Legal Perspectives} \cite{Custers2016}, considera-se que o primeiro uso de um drone registrado ocorreu em julho de 1849, quando as forças austríacas tentaram lançar balões incendiários com explosivos e uma bomba relógio para fazer os mesmo caírem sobre a cidade de Veneza. O drone como é conhecido hoje foi concebido por Abe Karem em 1977, quando criou um drone que era controlado por 3 pessoas, em vez de 30, como exigido pelos drones da época \cite{Buzzo2015}.

Hoje, os drones são muito populares, tanto para uso recreativo quanto para diversos outros fins. O Departamento de Controle do Espaço Aéreo através da Instrução de Comando da Força Aérea (ICA) 100-40 \cite{CEA2018} define os modelos usados apenas para fins recreativos como \textit{aeromodelos} e os outros modelos como \textit{Aeronaves Remotamente Pilotada - ARP}. Normalmente, os \textit{aeromodelos} são operados por humanos através de um controle remoto, enquanto as \textit{ARP} também podem ser automáticos ou autônomos. As \textit{ARPs} automáticas são modelos capazes de operar por conta própria e também podem ser controlados manualmente a qualquer momento, enquanto os autônomos têm seu caminho definido anteriormente e não podem ter intervenção humana durante a realização da missão.

Para que o drone possa executar uma missão automática ou autonomamente, é necessário um sistema de controle capaz de interpretar os dados dos sensores, fazer cálculos e tomar decisões. Alguns modelos de drones possuem sistemas operacionais e sensores embarcados que são suficientes para que se projete uma missão automática. Em outros casos, como o modelo \textit{Parrot AR.Drone 2.0} \cite{Parrot2019a}, utilizado pelo LaSid, o sistema é muito simples para ser capaz de realizar tarefas mais complexas. Uma solução é adotar um sistema auxiliar acoplado ao drone, capaz de interagir com as APIs do sistema do drone. Um sistema que se adequa bem a essa tarefa é o Raspberry Pi.

O Raspberry Pi é um computador do tamanho de um cartão de crédito que pode se conectar a uma TV ou monitor e usar periféricos como mouse e teclado \cite{RPF2019a}. O sistema operacional padrão é o Linux, cuja distribuição principal é a Raspbian, com base na distribuição Debian. A possibilidade de usar o Linux torna o Raspberry uma solução muito mais flexível em comparação com outras soluções de microcontroladores, como o Arduino, por exemplo.

A desvantagem vem do fato de que sistemas como o Arduino podem ser determinísticos, ou seja, sempre respondem dentro de um tempo máximo calculado, enquanto o sistema Linux padrão não tem garantias temporais para responder a um evento externo. No entanto a frequência de operação do processador do Raspberry é muito superior à do Arduino (1200MHz no modelo utilizado pelo LaSid contra 16 MHz do Arduino Uno, modelo mais comum), o Raspberry quase sempre pode responder mais rápido do que uma solução semelhante no Arduino.

Para a maioria das aplicações, este comportamento pode ser satisfatório. Em aplicações críticas, no entanto, uma única resposta atrasada pode significar uma falha catastrófica. Para lidar com esses cenários em que o tempo máximo de resposta deve ser previsível, foram criados patches para o kernel Linux a fim de torná-lo um Sistema de Tempo Real. A definição de Sistema de Tempo Real é "um sistema de tempo real é aquele que deve satisfazer explicitamente restrições de tempo de resposta podendo ter consequências de risco ou falha não satisfazendo às suas restrições"  \cite{Laplante2004}.

\section{Motivação}
\section{Objetivos}
\section{Organização}


