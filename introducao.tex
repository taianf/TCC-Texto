\xchapter{Introdução}{}

 % A história dos drones pode parecer recente diante dos avanços tecnológicos que baratearam as tecnologias envolvidas nos modelos mais conhecidos, mas, ao buscar sua origem, percebemos que estamos usando drones há mais de um século. No livro \textit{The Future of Drone Use: Opportunities and Threats from Ethical and Legal Perspectives} \cite{Custers2016}, considera-se que o primeiro uso de um drone registrado ocorreu em julho de 1849, quando as forças austríacas tentaram lançar balões incendiários com explosivos e uma bomba relógio para fazer os mesmo caírem sobre a cidade de Veneza. O drone como é conhecido hoje foi concebido por Abe Karem em 1977, quando criou um drone que era controlado por 3 pessoas, em vez de 30, como exigido pelos drones da época \cite{Buzzo2015}.

% \section{Apresentação}

Sistemas computacionais modernos possuem cada vez mais requisitos e complexidade para atender o seu crescente número de aplicações. Alguns sistemas possuem requisitos temporais, estes são chamados de sistemas de tempo real. Um Sistema de Tempo Real (STR) tem que garantir que a resposta, além de correta, será entregue a tempo, de modo a garantir que este sistema seja previsível e possa ter garantias temporais. Sistemas embarcados como carros e aeronaves, plantas industriais e sistemas robóticos, são apenas alguns exemplos de sistemas que precisam de uma resposta correta e em tempo restrito para que a aplicação funcione corretamente.

Para termos um Sistema de Tempo Real, é necessário que o sistema operacional esteja preparado para lidar com esse tipo de aplicação específica, pois, com a evolução tecnológica do hardware, surgem novos fatores que podem trazer variações no tempo de resposta do sistema. Por exemplo, temos os adventos de memória cache, acesso direto à memória, co-processamento, predição de instruções, unidades multicore, pipelines e execução fora de ordem, que constituem fontes de indeterminismo \cite{Liu2000, Pratt2004}. Todos esses fatores são desafios a serem resolvidos para permitir que Sistema Operacional de Propósito Geral possa dar garantias temporais similares a um Sistema Operacional de Tempo Real (SOTR) dedicado.

\section{Motivação}

Hoje, os drones são muito populares, tanto para uso recreativo quanto para diversos outros fins. O Departamento de Controle do Espaço Aéreo Brasileiro através da Instrução de Comando da Força Aérea (ICA) 100-40 \cite{CEA2018} define os modelos usados apenas para fins recreativos como \textit{aeromodelos} e os outros modelos como \textit{Aeronaves Remotamente Pilotada - ARP}. Normalmente, os \textit{aeromodelos} são operados por humanos através de um controle remoto, enquanto as \textit{ARP} também podem ser automáticos ou autônomos. As \textit{ARPs} automáticas são modelos capazes de operar por conta própria e também podem ser controlados manualmente a qualquer momento, enquanto os autônomos têm seu caminho definido anteriormente e não podem ter intervenção humana durante a realização da missão.

Para que o drone possa executar uma missão automática ou autonomamente, é necessário um sistema de controle capaz de interpretar os dados dos sensores, fazer cálculos e tomar decisões. Alguns modelos de drones possuem sistemas operacionais e sensores embarcados que são suficientes para que se projete uma missão automática. Em outros casos, como o modelo \textit{Parrot AR.Drone 2.0} \cite{Parrot2019a}, utilizado pelo LaSid, o sistema é muito simples para ser capaz de realizar tarefas mais complexas. Uma solução é adotar um sistema auxiliar acoplado ao drone, capaz de interagir com as APIs do sistema do drone. Um sistema que se adapta bem a essa tarefa é o Raspberry Pi.

O Raspberry Pi é um computador do tamanho de um cartão de crédito que pode se conectar a uma TV ou monitor e usar periféricos como mouse e teclado \cite{RPF2019}. O sistema operacional padrão é o Linux, cuja distribuição principal é a Raspbian, com base na distribuição Debian. A possibilidade de usar o Linux torna o Raspberry Pi uma solução muito mais flexível em comparação com outras soluções de microcontroladores, como o Arduino, por exemplo.

A desvantagem vem do fato de que sistemas como o Arduino podem ser determinísticos, ou seja, sempre responder dentro de um tempo máximo calculado, enquanto o sistema Linux padrão não tem garantias temporais para responder a um evento externo. No entanto a frequência de operação do processador do Raspberry Pi é muito superior à do Arduino (1200 MHz no modelo utilizado pelo LaSid contra 16 MHz do Arduino Uno, modelo mais comum), e portanto, uma solução baseada no Raspberry Pi quase sempre pode responder mais rápido do que uma solução semelhante baseada no Arduino.

Na maioria das aplicações, o comportamento do Raspberry Pi pode ser satisfatório. Em aplicações críticas, no entanto, uma única resposta atrasada pode significar uma falha catastrófica. Para lidar com esses cenários em que o tempo máximo de resposta deve ser previsível, foram criados patches para o kernel Linux a fim de torná-lo um Sistema Operacional de Tempo Real. 

Durante a realização deste trabalho, não foi encontrado nenhum trabalho que verifique o determinismo do Raspberry Pi levando em conta os diferentes tipos de implementação do sistema operacional Linux e o impacto das interrupções nas garantias temporais. As análises da latência de interrupção costumam ser testes de caixa preta: testes onde o funcionamento interno é desconhecido, sendo medido apenas o tempo entre o sistema receber um estímulo externo que requer a atenção do sistema e o mesmo responder a esse estímulo com uma resposta visível; em contraste com testes de caixa branca: testes onde é analisado o funcionamento interno do objeto sendo medido. Ao fazer uma análise em mais baixo nível, isto é, levando em conta os mecanismos de interrupção presentes no kernel e o comportamento diferente deles, é possível separar o tempo de resposta do sistema do tempo de resposta da aplicação.

\section{Objetivos}

\subsection{Objetivo principal}

O objetivo deste trabalho é verificar o determinismo da plataforma Raspberry Pi 3 Model B dotado do sistema operacional Linux padrão, Linux Preempt-RT e Linux com o Xenomai em diferentes cenários de execução de aplicações com requisitos temporais. Para este feito, realiza-se experimentos para quantificar as diversas latências do sistema em vários cenários de execução, de forma a ajudar no desenvolvimento de aplicações com requisitos temporais nesta plataforma. Para realizar os experimentos foi o usado o módulo INTSight, que é um módulo kernel para realizar benchmarks das latências dos diferentes mecanismos de interrupção desenvolvido por Luis Gerhorst \cite{Gerhorst2018}, que será apresentado na seção \ref{INTSight}. Espera-se que esses experimentos ajudem na decisão do tipo de plataforma e de sistema a ser utilizado de acordo com os requisitos da aplicação desejada.

\subsection{Objetivos intermediários}

O objetivo principal se desdobrou nos seguintes objetivos intermediários:

\begin{itemize}
    \item Adaptar o INTSight para o Raspberry Pi;
    \item Adaptar o INTSight no kernel padrão do Linux e realizar as medições;
    \item Adaptar o INTSight no patch Preempt-RT e realizar as medições;
    \item Adaptar o INTSight no patch Xenomai e realizar as medições;
    \item Analisar o resultado obtido nas medições;
\end{itemize}

Destes objetivos intermediários, a adaptação do Xenomai não foi cumprida por este trabalho. As dificuldades estão relatadas na seção \ref{Dificuldades}.

\section{Organização}

No capítulo \ref{cap2} irei apresentar alguns trabalhos relacionados e quais as estratégias utilizadas pelos autores em cada um para realizar suas medições ou análises. A descrição dos modelos de SO utilizados nas medições além de características do Raspberry Pi Model B serão apresentadas no capítulo \ref{cap3}. A apresentação e análise de resultados experimentais está no capítulo \ref{cap4}, onde faço uma comparação entre os modelos de interrupção e os SO's. Na última parte, no capítulo \ref{cap5}, serão apresentadas as dificuldades encontradas durante a realização deste trabalho e ideias para continuação do trabalho no futuro.
