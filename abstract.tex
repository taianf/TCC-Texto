\abstract
The popularization of drones has brought practicality and convenience to perform various tasks that without their help would be very laborious or even very dangerous. Some simpler drones lack the ability to add more sensors and a common alternative is to couple an embedded system, such as a Raspberry Pi, to control the drone through the drone manufacturer's API. This embedded system must always be able to respond within a critical time frame, as a delay in sensor reading or drone control, depending on its application, can lead to catastrophic failures and potentially endanger human life. Considering these systems for their timing is not a trivial task because of the amount of software and hardware variables that can influence response time. The most appropriate way to evaluate in this case is to conduct a batch of measurements to evaluate how the system responds. In this context, this work aims to evaluate the response time of Raspberry Pi 3 Model B, model used in LaSiD - Laboratório de Sistemas Distribuídos for research with the AR.Drone 2.0 drone manufactured by Parrot, in different scenarios and in different implementations of the Linux kernel, such as the PREEMPT-RT patch, to help application designers develop drone control systems.

% Palavras-chave do resumo em Ingles
\begin{keywords}
Raspberry Pi; Real-time systems; PREEMPT-RT; Drones; Embedded systems
\end{keywords}
