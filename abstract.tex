\abstract
The popularization of drones has brought practicality and convenience to perform various tasks that without their help would be very laborious or even very dangerous. According to ANAC, Agência Nacional de Aviação Civil, in July 2019 Brazil already had more than 70 thousand registered drones, more than 25 thousand for professional use and more than 45 thousand for recreational use. For some simpler activities manual control of a drone is sufficient but for more extensive or repetitive activities manual control is not appropriate. Some simpler drones lack the ability to add more sensors and a common alternative is to couple an embedded system, such as a Raspberry Pi, to control the drone through the drone manufacturer's API. This embedded system must always be able to respond within a critical time frame, as a delay in sensor reading or drone control, depending on its application, can lead to catastrophic failures and potentially endanger human life. In this context, this work aims to evaluate the response time of Raspberry Pi 3 Model B, model used in LaSiD - Laboratório de Sistemas Distribuídos for research with the AR.Drone 2.0 drone manufactured by Parrot, in different scenarios and in different implementations of the Linux kernel, such as the PREEMPT-RT patch, to help application designers develop drone control systems.

% Palavras-chave do resumo em Ingles
\begin{keywords}
Raspberry Pi; Real-time systems; PREEMPT-RT; Drones; Embedded systems
\end{keywords}
