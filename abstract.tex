\abstract
Modern computer systems are capable of performing many tasks and also need to meet a variety of requirements. Some systems have time requirements and need to respond quickly and correctly to external events such as cars, aircraft, robotic systems, and other mechatronic systems, where a delayed response may compromise system integrity. For this, specific operating systems are designed to meet these requirements, the real-time operating systems. These external events are signaled to the processor through the interrupt mechanism. To keep the system responsive, these interrupts need to be addressed in a short and predictable time. As systems are increasingly complex, accurate analysis models do not exist, so measurement strategies are the best way to verify system responsiveness. One of the most popular embedded systems is the Raspberry Pi, a small, low-cost, low-power platform that uses Linux, one of the most popular operating systems in the world. This makes Raspberry Pi one of the most widely used platforms for the development of embedded systems in various projects. This work analyzes Raspberry's latency when responding to external events on standard Linux and Preempt-RT, a realtime patch for Linux.

% Palavras-chave do resumo em Ingles
\begin{keywords}
Raspberry Pi; Real-time systems; Preempt-RT; Embedded systems
\end{keywords}
