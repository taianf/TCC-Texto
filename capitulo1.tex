\xchapter{Introdução}{Opcional}

A história dos drones pode parecer recente diante dos avanços tecnológicos que baratearam as tecnologias envolvidas nos modelos mais conhecidos, mas, ao buscar sua origem, percebemos que estamos usando drones há mais de um século. Em \textit{The Future of Drone Use: Opportunities and Threats from Ethical and Legal Perspectives} \cite{Custers2016}, considera-se que o primeiro uso de um drone registrado ocorreu em julho de 1849, quando as forças austríacas tentaram lançar balões incendiários com explosivos e uma bomba relógio para fazer os mesmo caírem sobre a cidade de Veneza. O drone como é conhecido hoje foi concebido por Abe Karem em 1977, quando criou um drone que era controlado por 3 pessoas, em vez de 30, como exigido pelos drones da época \cite{Buzzo2015}.

Hoje, os drones são muito populares, tanto para uso recreativo quanto para diversos outros fins. O Departamento de Controle do Espaço Aéreo através da Instrução de Comando da Força Aérea (ICA) 100-40 \cite{CEA2018} define os modelos usados apenas para fins recreativos como \textit{aeromodelos} e os outros modelos como \textit{Aeronaves Remotamente Pilotada - ARP}. Normalmente, os \textit{aeromodelos} são operados por humanos através de um controle remoto, enquanto o \textit{ARP} também podem ser automáticos ou autônomos. Os ARPs automáticos são modelos capazes de operar por conta própria e também podem ser controlados manualmente a qualquer momento, enquanto os autônomos têm seu caminho definido anteriormente e não podem ter intervenção humana durante a realização da missão.

Para

\section{Motivação}
\section{Objetivos}
\section{Organização}
\section{Objetivos}
