\xchapter{Introdução}{Opcional}

Apresentação do trabalho. Teste de citação. \cite{Sousa2017}

Outro parágrafo.

A história dos drones pode parecer recente frente aos avanços tecnológicos que baratearam as tecnologias envolvidas nos modelos mais conhecidos, mas ao buscar a sua origem vemos usos há mais de um século. No livro \textit{The Future of Drone Use: Opportunities and Threats from Ethical and Legal Perspectives} \cite{Custers2016} é considerado que o primeiro uso de um drone registrado ocorreu em julho de 1849 quando forças austríacas tentaram lançar balões incendiários contendo explosivos e uma bomba relógio para fazer os mesmos caírem sobre a cidade. O drone como é conhecido hoje foi concebido por Abe Karem em 1977, quando ele criou um drone que era controlado por 3 pessoas, ao invés de 30, como os drones da época exigiam \cite{Buzzo2015}.

Os drones são muito populares atualmente tanto para uso recreativo quando para fins diversos. O Departamento de Controle do Espaço Aéreo através da Instrução do Comando da Aeronáutica (ICA) 100-40 define os modelos utilizados com propósitos apenas recreativos como Aeromodelos e os demais modelos como Aeronaves Remotamente Pilotadas - ARP. Os aeromodelos são normalmente operados por um humano através de um controle remoto enquanto as ARP podem também serem automáticas ou autônomas. As ARP automáticas são os modelos capazes de operar sozinhas podendo também ser controladas manualmente a qualquer momento enquanto as autônomas tem seu trajeto definido previamente e não podem ter intervenção humana durante a realização da missão. \cite{CEA2018}



Drones na agricultura.
Acidentes com drones.
Automáticos.
Semi-atuomáticos.
Manuais.

\section{Motivação}
\section{Objetivos}
\section{Organização}
\section{Objetivos}
