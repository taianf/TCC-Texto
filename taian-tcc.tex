%% Template para disserta????o/tese na classe UFBAthesis
%% versao 1.0
%% (c) 2005 Paulo G. S. Fonseca
%% (c) 2012 Antonio Terceiro
%% (c) 2014 Christina von Flach
%% www.dcc.ufba.br/~flach/ufbathesis

%% Carrega a classe ufbathesis
%% Opcoes: * Idiomas
%%           pt   - portugues (padrao)
%%           en   - ingles
%%         * Tipo do Texto
%%           bsc  - para monografias de graduacao
%%           msc  - para dissertacoes de mestrado (padrao)
%%           qual - exame de qualificacao de mestrado
%%           prop - exame de qualificacao de doutorado
%%           phd  - para teses de doutorado
%%         * M??dia
%%           scr  - para vers??o eletr??nica (PDF) / consulte o guia do usuario
%%         * Estilo
%%           classic - estilo original a la TAOCP (deprecated)
%%           std     - novo estilo a la CUP (padrao)
%%         * Paginacao
%%           oneside - para impressao em face unica
%%           twoside - para impressao em frente e verso (padrao)
\documentclass[bsc, classic, a4paper]{ufbathesis}
\usepackage[utf8]{inputenc}

%% Preambulo:
%% coloque aqui o seu preambulo LaTeX, i.e., declara????o de pacotes,
%% (re)definicoes de macros, medidas, etc.

%% Identificacao:

% Universidade
\university{UNIVERSIDADE FEDERAL DA BAHIA}

% Endereco (cidade)
% e.g. \address{Campinas}
\address{Salvador}

% Instituto ou Centro Academico
% e.g. \institute{Centro de Ciencias Exatas e da Natureza}
% Comente se nao se aplicar
\institute{Departamento de Engenharia Elétrica e Computação}

% Nome da biblioteca - usado na ficha catalografica
% default: nome da biblioteca do Instituto de Matematica
\library{BIBLIOTECA REITOR MACÊDO COSTA}

% Programa de pos-graduacao
% e.g. \program{Pos-graduacao em Ciencia da Computacao}
\program{Departamento de Engenharia Elétrica e de Computação}

% ?rea de titulacao
\majorfield{Engenharia de Computação}

% Titulo da dissertacao/tese
% e.g. \title{Sobre a conjectura $P=NP$}
\title{Análise do tempo da latência de interrupção no Raspberry Pi em diferentes patches de tempo real para o kernel Linux}

% Data da defesa
% e.g. \date{19 de fevereiro de 2003}
\date{DIA de MÊS de ANO}

% Autor
% e.g. \author{Jose da Silva}
\author{Taian Fonseca Feitosa}

% Orientador(a)
% Opcao: [f] - para orientador do sexo feminino
% e.g. \adviser[f]{Profa. Dra. Maria Santos}
\adviser{Paul Regnier}

% Orientador(a)
% Opcao: [f] - para orientador do sexo feminino
% e.g. \coadviser{Prof. Dr. Pedro Pedreira}
% Comente se nao se aplicar
% \coadviser{NOME DO(DA) CO-ORIENTADOR(A)}

%% Inicio do documento
\begin{document}

\pgcompfrontpage{????? PGCOMP-Bsc-XXXX}

%%
%% Parte pre-textual
%%
\frontmatter

%\pgcomppresentationpage
% Se seu trabalho for uma disserta??o de mestrado do PGCOMP, use a linha acima
%\presentationpage
% Se for qualificacao, use \presentationpage

% Ficha catalogrofica
\authorcitationname{Feitosa, Taian Fonseca} % e.g. Terceiro, Antonio Soares de Azevedo
\advisercitationname{Regnier, Paul} % e.g. Chavez, Christina von Flach Garcia
%\coadvisercitationname{NOME DO SEU CO-ORIENTADOR EM CITACOES} % e.g. Mendonca, Manoel Gomes de
\catalogtype{Trabalho de Conclusão de Curso} % e.g. ``Tese (doutorado)''
\catalogtopics{1. Raspberry Pi. 2. Sistemas de tempo real. 3. PREEMPT-RT. 4. Drones. 5. Sistemas embarcados} % e.g. ``1. Complexidade Estrutural. 2. Engenharia de Software''
\catalogcdd{CDD 20.ed. XXX.YY} % e.g. ``CDD 20.ed. XXX.YY'' (esse n??mero vai lhe ser dado pela biblioteca)
\catalogingsheet

% Termo de aprovacaoo
% Se seu trabalho for Tese de Doutorado, inclua comitteemember 4 e 5
\approvalsheet{Salvador, DIA de MES de ANO}{
  \comittemember{Profa. Dra. Professora 1}{Universidade XYZ}
  \comittemember{Prof. Dr. Professor 2}{Universidade 123}
  \comittemember{Profa. Dra. Professora 3}{Universidade ABC}
%  \comittemember{Prof. Dr. Professor 4}{Universidade HJKL}
%  \comittemember{Profa. Dra. Professora 5}{Universidade QWERTY}
}

% Dedicatoria
% Comente para ocultar
\begin{dedicatory}
Dedicatória
\end{dedicatory}

% Agradecimentos
% Se preferir, crie um arquivo ?? parte e o inclua via \include{}
\acknowledgements
Agradecimentos

% Epigrafe
% Comente para ocultar
% e.g.
\begin{epigraph}[Um dia]{Eu mesmo}
Epígrafe
\end{epigraph}

% Resumo em Portugues
% Se preferir, crie um arquivo separado e o inclua via \include{}
\resumo
A popularização dos Veículos Aéreos Não Tripulados (VANTs), mais conhecidos como drones, trouxe praticidade e comodidade para executar várias tarefas que sem sua ajuda seriam muito trabalhosas ou mesmo muito perigosas. Segunda a ANAC, Agência Nacional de Aviação Civil, em julho de 2019 o Brasil já tinha mais de 70 mil drones cadastrados, sendo mais de 25 mil para uso profissional e mais de 45 mil para uso recreativo. Para algumas atividades mais simples o controle manual de um drone é suficiente mas para atividades mais extensivas ou repetitivas o controle manual não é apropriado. Alguns drones mais simples não possuem capacidade de adicionar mais sensores e uma alternativa comum é acoplar um sistema embarcado, como um Raspberry Pi, para controlar o drone através da API do fabricante do drone. Este sistema embarcado deve sempre ser capaz de responder dentro de um período crítico, pois um atraso na leitura do sensor ou no controle do drone, dependendo de sua aplicação, pode levar a falhas catastróficas e potencialmente colocar em risco a vida humana. Neste contexto, este trabalho tem como objetivo avaliar o tempo de resposta de Raspberry Pi 3 Modelo B, modelo utilizado no LaSiD - Laboratório de Sistemas Distribuídos para pesquisas com o drone AR.Drone 2.0 fabricado pela Parrot, em diferentes cenários e em diferentes implementações do kernel do linux, como o patch PREEMPT-RT, para ajudar os projetistas de aplicativos a desenvolver sistemas de controle de drones.

% Palavras-chave do resumo em Portugues
\begin{keywords}
Raspberry Pi; Sistemas de tempo real; PREEMPT-RT; Drones; Sistemas embarcados
\end{keywords}


% Resumo em Ingles
% Se preferir, crie um arquivo separado e o inclua via \include{}
\abstract
Modern computer systems are capable of performing many tasks and also need to meet a variety of requirements. Some systems have time requirements and need to respond quickly and correctly to external events such as cars, aircraft, robotic systems, and other mechatronic systems, where a delayed response may compromise system integrity. For this, specific operating systems are designed to meet these requirements, the real-time operating systems. These external events are signaled to the processor through the interrupt mechanism. To keep the system responsive, these interrupts need to be addressed in a short and predictable time. As systems are increasingly complex, accurate analysis models do not exist, so measurement strategies are the best way to verify system responsiveness. One of the most popular embedded systems is the Raspberry Pi, a small, low-cost, low-power platform that uses Linux, one of the most popular operating systems in the world. This makes Raspberry Pi one of the most widely used platforms for the development of embedded systems in various projects. This work analyzes Raspberry's latency when responding to external events on standard Linux and Preempt-RT, a realtime patch for Linux.

% Palavras-chave do resumo em Ingles
\begin{keywords}
Raspberry Pi; Real-time systems; Preempt-RT; Embedded systems
\end{keywords}


% Sumario
% Comente para ocultar
\tableofcontents

% Lista de figuras
% Comente para ocultar
\listoffigures

% Lista de tabelas
% Comente para ocultar
\listoftables

%%
%% Parte textual
%%
\mainmatter

% Eh aconselhavel criar cada capitulo em um arquivo separado, digamos
% "capitulo1.tex", "capitulo2.tex", ... "capituloN.tex" e depois
% inclui-los com:
\xchapter{Introdução}{Opcional}

A história dos drones pode parecer recente diante dos avanços tecnológicos que baratearam as tecnologias envolvidas nos modelos mais conhecidos, mas, ao buscar sua origem, percebemos que estamos usando drones há mais de um século. Em \textit{The Future of Drone Use: Opportunities and Threats from Ethical and Legal Perspectives} \cite{Custers2016}, considera-se que o primeiro uso de um drone registrado ocorreu em julho de 1849, quando as forças austríacas tentaram lançar balões incendiários com explosivos e uma bomba relógio para fazer os mesmo caírem sobre a cidade de Veneza. O drone como é conhecido hoje foi concebido por Abe Karem em 1977, quando criou um drone que era controlado por 3 pessoas, em vez de 30, como exigido pelos drones da época \cite{Buzzo2015}.

Hoje, os drones são muito populares, tanto para uso recreativo quanto para diversos outros fins. O Departamento de Controle do Espaço Aéreo através da Instrução de Comando da Força Aérea (ICA) 100-40 \cite{CEA2018} define os modelos usados apenas para fins recreativos como \textit{aeromodelos} e os outros modelos como \textit{Aeronaves Remotamente Pilotada - ARP}. Normalmente, os \textit{aeromodelos} são operados por humanos através de um controle remoto, enquanto o \textit{ARP} também podem ser automáticos ou autônomos. Os ARPs automáticos são modelos capazes de operar por conta própria e também podem ser controlados manualmente a qualquer momento, enquanto os autônomos têm seu caminho definido anteriormente e não podem ter intervenção humana durante a realização da missão.

Para

\section{Motivação}
\section{Objetivos}
\section{Organização}
\section{Objetivos}

\xchapter{DRONES/VANTs}{Opcional}

Coisas sobre o drone.

Outro parágrafo.

\section{DRONES/VANTs}
\section{Quantidade}
\section{Usos}
\section{Arquitetura}

\xchapter{Raspberry}{Opcional}

Um pouco do raspberry.

Outro parágrafo.

\section{Modelos}
\section{Usos}
\section{Arquitetura}
\xchapter{Sistemas de tempo real}{Opcional}

Aqui o bagulho é na hora.

Outro parágrafo.

\section{Modelos}
\section{Hard}
\section{Patches}
\xchapter{Revisão}{Opcional}

Aqui o bagulho revisa.

Outro parágrafo.

\section{Quem fez o quê}
\section{Como fizeram}
\section{Mais algo?}
\xchapter{Avaliação}{Opcional}

Meus resultados.

Outro parágrafo.

\section{Como avaliei}
\section{Resultados esperados}
\section{Resultados obtidos}
\xchapter{Conclusão}{Opcional}

Aqui o bagulho acaba! É tetra!

Outro parágrafo.

\section{Aprendizados}
\section{Dificuldades}
\section{Pro futuro (?)}
% \include{capitulo8}
% ...
% \include{capituloN}
%
% Importante: Use \xchapter ao inves de \chapter, conforme exemplo abaixo.

%%
%% Parte pos-textual
%%
\backmatter

% Apendices
% Comente se n??o houver ap??ndices
\appendix

% Eh aconselhavel criar cada apendice em um arquivo separado, digamos
% "apendice1.tex", "apendice.tex", ... "apendiceM.tex" e depois
% inclui--los com:
% \include{apendice1}
% \include{apendice2}
% ...
% \include{apendiceM}

% Bibliografia
% ?? aconselh??vel utilizar o BibTeX a partir de um arquivo, digamos "biblio.bib".
% Para ajuda na cria????o do arquivo .bib e utiliza????o do BibTeX, recorra ao
% BibTeXpress em www.cin.ufpe.br/~paguso/bibtexpress
\bibliographystyle{abntex2-alf}
\bibliography{biblio}

% Colofon
% Inclui uma pequena nota com referencia a UFPEThesis
% Comente para omitir
%\colophon

%% Fim do documento
\end{document}
