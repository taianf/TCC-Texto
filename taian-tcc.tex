%% Template para disserta????o/tese na classe UFBAthesis
%% versao 1.0
%% (c) 2005 Paulo G. S. Fonseca
%% (c) 2012 Antonio Terceiro
%% (c) 2014 Christina von Flach
%% www.dcc.ufba.br/~flach/ufbathesis

%% Carrega a classe ufbathesis
%% Opcoes: * Idiomas
%%           pt   - portugues (padrao)
%%           en   - ingles
%%         * Tipo do Texto
%%           bsc  - para monografias de graduacao
%%           msc  - para dissertacoes de mestrado (padrao)
%%           qual - exame de qualificacao de mestrado
%%           prop - exame de qualificacao de doutorado
%%           phd  - para teses de doutorado
%%         * M??dia
%%           scr  - para vers??o eletr??nica (PDF) / consulte o guia do usuario
%%         * Estilo
%%           classic - estilo original a la TAOCP (deprecated)
%%           std     - novo estilo a la CUP (padrao)
%%         * Paginacao
%%           oneside - para impressao em face unica
%%           twoside - para impressao em frente e verso (padrao)
\documentclass[bsc, classic, a4paper]{ufbathesis}
\usepackage[utf8]{inputenc}

%% Preambulo:
%% coloque aqui o seu preambulo LaTeX, i.e., declara????o de pacotes,
%% (re)definicoes de macros, medidas, etc.

%% Identificacao:

% Universidade
\university{UNIVERSIDADE FEDERAL DA BAHIA}

% Endereco (cidade)
% e.g. \address{Campinas}
\address{Salvador}

% Instituto ou Centro Academico
% e.g. \institute{Centro de Ciencias Exatas e da Natureza}
% Comente se nao se aplicar
\institute{Departamento de Engenharia Elétrica e Computação}

% Nome da biblioteca - usado na ficha catalografica
% default: nome da biblioteca do Instituto de Matematica
\library{BIBLIOTECA REITOR MACÊDO COSTA}

% Programa de pos-graduacao
% e.g. \program{Pos-graduacao em Ciencia da Computacao}
\program{Departamento de Engenharia Elétrica e de Computação}

% ?rea de titulacao
\majorfield{Engenharia de Computação}

% Titulo da dissertacao/tese
% e.g. \title{Sobre a conjectura $P=NP$}
\title{Análise do tempo da latência de interrupção no Raspberry Pi em diferentes patches de tempo real para o kernel Linux}

% Data da defesa
% e.g. \date{19 de fevereiro de 2003}
\date{xx de Julho de 2019}

% Autor
% e.g. \author{Jose da Silva}
\author{Taian Fonseca Feitosa}

% Orientador(a)
% Opcao: [f] - para orientador do sexo feminino
% e.g. \adviser[f]{Profa. Dra. Maria Santos}
\adviser{Paul Regnier}

% Orientador(a)
% Opcao: [f] - para orientador do sexo feminino
% e.g. \coadviser{Prof. Dr. Pedro Pedreira}
% Comente se nao se aplicar
% \coadviser{NOME DO(DA) CO-ORIENTADOR(A)}

%% Inicio do documento
\begin{document}

\pgcompfrontpage{????? PGCOMP-Bsc-XXXX}

%%
%% Parte pre-textual
%%
\frontmatter

%\pgcomppresentationpage
% Se seu trabalho for uma disserta??o de mestrado do PGCOMP, use a linha acima
%\presentationpage
% Se for qualificacao, use \presentationpage

% Ficha catalogrofica
\authorcitationname{Feitosa, Taian Fonseca} % e.g. Terceiro, Antonio Soares de Azevedo
\advisercitationname{Regnier, Paul} % e.g. Chavez, Christina von Flach Garcia
%\coadvisercitationname{NOME DO SEU CO-ORIENTADOR EM CITACOES} % e.g. Mendonca, Manoel Gomes de
\catalogtype{Trabalho de Conclusão de Curso} % e.g. ``Tese (doutorado)''
\catalogtopics{TOPICOS PARA FICHA CATALOGRAFICA} % e.g. ``1. Complexidade Estrutural. 2. Engenharia de Software''
\catalogcdd{NUMERO CDD} % e.g. ``CDD 20.ed. XXX.YY'' (esse n??mero vai lhe ser dado pela biblioteca)
\catalogingsheet

% Termo de aprovacaoo
% Se seu trabalho for Tese de Doutorado, inclua comitteemember 4 e 5
\approvalsheet{Salvador, DIA de MES de ANO}{
  \comittemember{Profa. Dra. Professora 1}{Universidade XYZ}
  \comittemember{Prof. Dr. Professor 2}{Universidade 123}
  \comittemember{Profa. Dra. Professora 3}{Universidade ABC}
%  \comittemember{Prof. Dr. Professor 4}{Universidade HJKL}
%  \comittemember{Profa. Dra. Professora 5}{Universidade QWERTY}
}

% Dedicatoria
% Comente para ocultar
\begin{dedicatory}
DIGITE A DEDICATORIA AQUI
\end{dedicatory}

% Agradecimentos
% Se preferir, crie um arquivo ?? parte e o inclua via \include{}
\acknowledgements
DIGITE OS AGRADECIMENTOS AQUI

% Epigrafe
% Comente para ocultar
% e.g.
%  \begin{epigraph}[Tarde, 1919]{Olavo Bilac}
%  ??ltima flor do L??cio, inculta e bela,\\
%  ??s, a um tempo, esplendor e sepultura;\\
%  Ouro nativo, que, na ganga impura,\\
%  A bruta mina entre os cascalhos vela.
%  \end{epigraph}
\begin{epigraph}[NOTA]{AUTOR}
DIGITE AQUI A CITACAO
\end{epigraph}

% Resumo em Portugues
% Se preferir, crie um arquivo separado e o inclua via \include{}
\resumo
O aumento da popularidade dos drones tem trazido praticidade e comodidade para a realização de várias tarefas que sem a ajuda dos mesmos seriam muito trabalhosas ou mesmo muito perigosas. O controle automático dos drones exige um sistema de tempo real, ou seja, que seja capaz de executar tarefas em um tempo crítico pois uma resposta atrasada pode ser tão prejudicial quanto uma resposta errada. A escolha e/ou construção desse sistema não é uma tarefa simples pois a avaliação do tempo de resposta desses sistemas envolve variáveis físicas que não são medidas com precisão através de software. Este trabalho mostra algumas opções existentes, e suas avaliações em termo de tempo de resposta, para implementação do sistema de tempo real em uma plataforma Raspberry Pi, que pode ser facilmente acoplada a um drone. [Resultados e conclusões]

% Palavras-chave do resumo em Portugues
\begin{keywords}
XYZ
\end{keywords}

% Resumo em Ingles
% Se preferir, crie um arquivo separado e o inclua via \include{}
\abstract
YYYY
% Palavras-chave do resumo em Ingles
\begin{keywords}
DIGITE AS PALAVRAS-CHAVE AQUI
\end{keywords}

% Sumario
% Comente para ocultar
\tableofcontents

% Lista de figuras
% Comente para ocultar
\listoffigures

% Lista de tabelas
% Comente para ocultar
\listoftables

%%
%% Parte textual
%%
\mainmatter

% Eh aconselhavel criar cada capitulo em um arquivo separado, digamos
% "capitulo1.tex", "capitulo2.tex", ... "capituloN.tex" e depois
% inclui-los com:
% \xchapter{Introdução}{Opcional}

A história dos drones pode parecer recente diante dos avanços tecnológicos que baratearam as tecnologias envolvidas nos modelos mais conhecidos, mas, ao buscar sua origem, percebemos que estamos usando drones há mais de um século. Em \textit{The Future of Drone Use: Opportunities and Threats from Ethical and Legal Perspectives} \cite{Custers2016}, considera-se que o primeiro uso de um drone registrado ocorreu em julho de 1849, quando as forças austríacas tentaram lançar balões incendiários com explosivos e uma bomba relógio para fazer os mesmo caírem sobre a cidade de Veneza. O drone como é conhecido hoje foi concebido por Abe Karem em 1977, quando criou um drone que era controlado por 3 pessoas, em vez de 30, como exigido pelos drones da época \cite{Buzzo2015}.

Hoje, os drones são muito populares, tanto para uso recreativo quanto para diversos outros fins. O Departamento de Controle do Espaço Aéreo através da Instrução de Comando da Força Aérea (ICA) 100-40 \cite{CEA2018} define os modelos usados apenas para fins recreativos como \textit{aeromodelos} e os outros modelos como \textit{Aeronaves Remotamente Pilotada - ARP}. Normalmente, os \textit{aeromodelos} são operados por humanos através de um controle remoto, enquanto o \textit{ARP} também podem ser automáticos ou autônomos. Os ARPs automáticos são modelos capazes de operar por conta própria e também podem ser controlados manualmente a qualquer momento, enquanto os autônomos têm seu caminho definido anteriormente e não podem ter intervenção humana durante a realização da missão.

Para

\section{Motivação}
\section{Objetivos}
\section{Organização}
\section{Objetivos}

% \xchapter{DRONES/VANTs}{Opcional}

Coisas sobre o drone.

Outro parágrafo.

\section{DRONES/VANTs}
\section{Quantidade}
\section{Usos}
\section{Arquitetura}

% ...
% \include{capituloN}
%
% Importante: Use \xchapter ao inves de \chapter, conforme exemplo abaixo.

\xchapter{Introdu\c{c}\~{a}o}{Este eh o primeiro cap\'{\i}tulo, onde eu conto toda a historia deste trabalho}

Lorem ipsum dolor sit amet, consectetur adipiscing elit. Praesent ac tellus turpis. Donec vitae lorem odio. Sed luctus vestibulum libero eget pellentesque. Vivamus in mi turpis. Donec molestie feugiat sollicitudin. Nam et eleifend tortor. Cras lacinia, magna in tristique consequat, urna lorem placerat odio, id viverra mi nulla nec sem. Ut imperdiet, felis a hendrerit imperdiet, velit metus eleifend diam, ut convallis neque augue vitae leo. Morbi condimentum rhoncus faucibus. Fusce elit justo, semper lobortis blandit sed, lacinia et ligula. Pellentesque elit magna, vestibulum vitae adipiscing in, tincidunt eu erat. Nullam eu lectus vel nunc eleifend consectetur. Phasellus faucibus blandit nisi, eget fermentum erat tincidunt et. Aliquam ullamcorper varius nunc, nec eleifend dolor porttitor eu. Etiam eget nunc eu erat facilisis consequat sed accumsan urna. Nam vitae eros et justo iaculis gravida in sed nisl. Donec auctor gravida ipsum. Morbi sed ipsum tortor, quis ornare purus.

Nulla facilisi. Praesent nec massa nulla, interdum lobortis orci. Integer convallis porttitor congue. Etiam lectus ipsum, iaculis eu malesuada malesuada, scelerisque semper est. In convallis ipsum lacus, at dapibus velit. Praesent semper diam eget eros ultrices quis elementum libero cursus. Aliquam erat volutpat. Nunc a lectus augue, id sagittis neque. Aliquam dignissim tempus sem ut pellentesque. Nulla libero orci, elementum vel malesuada mattis, adipiscing sed lorem. Cras lacinia elit vel magna mollis ut congue elit faucibus. Duis ornare mollis est, quis hendrerit neque convallis in.

Donec at leo sed nisl varius iaculis. Sed ullamcorper nisl sit amet metus sollicitudin quis cursus elit imperdiet. Donec vel ante sed elit tempor hendrerit. Quisque commodo neque a ante tempor rhoncus. Proin purus metus, interdum mattis elementum quis, mollis et felis. Curabitur id nibh ac sapien porta pellentesque vel nec sapien. Duis ullamcorper consectetur elit, eleifend tincidunt leo adipiscing a. Cras nunc quam, sodales sit amet semper at, pharetra quis libero. Vestibulum posuere risus sed ante interdum tempor. Sed nec enim lorem, eu ultrices nisl. Maecenas cursus lacus non metus convallis cursus. Cras egestas feugiat libero, quis luctus justo mattis ut. Proin vitae magna et purus convallis blandit ut sed risus. Nulla vel ligula massa, vel interdum tortor. Integer sit amet eros sit amet lectus suscipit adipiscing. Aenean sagittis diam eu orci porttitor luctus.

Sed sed elit eu ante malesuada aliquet. Nunc quis quam turpis, accumsan faucibus sapien. Nam eget sem velit, id eleifend elit. Phasellus auctor egestas libero a adipiscing. Integer convallis, nunc sed tincidunt sollicitudin, augue diam accumsan risus, sit amet lacinia neque quam at lorem. Quisque ac rhoncus mauris. Donec felis odio, consequat vitae pretium ullamcorper, blandit id lorem. Morbi nec tellus vitae velit eleifend bibendum at vel eros. Sed quis enim urna. Cras placerat ultricies ipsum, sed tempor lorem placerat vitae. Morbi tellus leo, blandit eget feugiat nec, bibendum egestas erat. Mauris lobortis congue vestibulum. Nunc ultricies arcu in neque hendrerit convallis. Phasellus facilisis dictum nulla vel iaculis. Donec in adipiscing tellus. Aliquam non turpis neque. Nam in velit nec tortor gravida convallis. Vivamus tincidunt eros quis elit ultricies at adipiscing urna tempor. Donec posuere, ante pulvinar laoreet adipiscing, eros nulla pharetra nulla, vel rutrum elit est vitae ligula.

Phasellus urna nulla, laoreet in imperdiet a, porttitor eu nibh. Phasellus ac arcu magna, sit amet dapibus eros. Maecenas eget metus vitae leo feugiat lobortis ac quis tortor. Pellentesque habitant morbi tristique senectus et netus et malesuada fames ac turpis egestas. Pellentesque vel augue metus. Quisque fringilla, arcu ut fringilla consectetur, mauris tellus euismod sem, in vehicula est leo vel nisi. Praesent sapien ligula, malesuada eget hendrerit ac, aliquet nec turpis. Cum sociis natoque penatibus et magnis dis parturient montes, nascetur ridiculus mus. Nulla facilisi. Mauris commodo elementum elit, fermentum placerat nulla iaculis nec. Vestibulum neque dui, cursus at pharetra eu, sollicitudin eget justo. In id dignissim augue. Fusce interdum malesuada ipsum, non aliquet quam auctor eget. Curabitur turpis libero, pharetra ut cursus porta, pellentesque vitae magna. Cras porttitor interdum nisl.

Lorem ipsum dolor sit amet, consectetur adipiscing elit. Praesent ac tellus turpis. Donec vitae lorem odio. Sed luctus vestibulum libero eget pellentesque. Vivamus in mi turpis. Donec molestie feugiat sollicitudin. Nam et eleifend tortor. Cras lacinia, magna in tristique consequat, urna lorem placerat odio, id viverra mi nulla nec sem. Ut imperdiet, felis a hendrerit imperdiet, velit metus eleifend diam, ut convallis neque augue vitae leo. Morbi condimentum rhoncus faucibus. Fusce elit justo, semper lobortis blandit sed, lacinia et ligula. Pellentesque elit magna, vestibulum vitae adipiscing in, tincidunt eu erat. Nullam eu lectus vel nunc eleifend consectetur. Phasellus faucibus blandit nisi, eget fermentum erat tincidunt et. Aliquam ullamcorper varius nunc, nec eleifend dolor porttitor eu. Etiam eget nunc eu erat facilisis consequat sed accumsan urna. Nam vitae eros et justo iaculis gravida in sed nisl. Donec auctor gravida ipsum. Morbi sed ipsum tortor, quis ornare purus.

\xchapter{Revis\~{a}o Bibliogr\'{a}fica}{Neste cap\'{\i}tulo eu apresento todo o material que eu estudei durante a elabora\c{c}\~{a}o do trabalho.}


Lorem ipsum dolor sit amet, consectetur adipiscing elit. Praesent ac tellus turpis. Donec vitae lorem odio. Sed luctus vestibulum libero eget pellentesque. Vivamus in mi turpis. Donec molestie feugiat sollicitudin. Nam et eleifend tortor. Cras lacinia, magna in tristique consequat, urna lorem placerat odio, id viverra mi nulla nec sem. Ut imperdiet, felis a hendrerit imperdiet, velit metus eleifend diam, ut convallis neque augue vitae leo. Morbi condimentum rhoncus faucibus. Fusce elit justo, semper lobortis blandit sed, lacinia et ligula. Pellentesque elit magna, vestibulum vitae adipiscing in, tincidunt eu erat. Nullam eu lectus vel nunc eleifend consectetur. Phasellus faucibus blandit nisi, eget fermentum erat tincidunt et. Aliquam ullamcorper varius nunc, nec eleifend dolor porttitor eu. Etiam eget nunc eu erat facilisis consequat sed accumsan urna. Nam vitae eros et justo iaculis gravida in sed nisl. Donec auctor gravida ipsum. Morbi sed ipsum tortor, quis ornare purus \cite{demeyer2008}.

Nulla facilisi. Praesent nec massa nulla, interdum lobortis orci. Integer convallis porttitor congue. Etiam lectus ipsum, iaculis eu malesuada malesuada, scelerisque semper est. In convallis ipsum lacus, at dapibus velit. Praesent semper diam eget eros ultrices quis elementum libero cursus. Aliquam erat volutpat. Nunc a lectus augue, id sagittis neque. Aliquam dignissim tempus sem ut pellentesque. Nulla libero orci, elementum vel malesuada mattis, adipiscing sed lorem. Cras lacinia elit vel magna mollis ut congue elit faucibus. Duis ornare mollis est, quis hendrerit neque convallis in.

Donec at leo sed nisl varius iaculis. Sed ullamcorper nisl sit amet metus sollicitudin quis cursus elit imperdiet. Donec vel ante sed elit tempor hendrerit. Quisque commodo neque a ante tempor rhoncus. Proin purus metus, interdum mattis elementum quis, mollis et felis. Curabitur id nibh ac sapien porta pellentesque vel nec sapien. Duis ullamcorper consectetur elit, eleifend tincidunt leo adipiscing a. Cras nunc quam, sodales sit amet semper at, pharetra quis libero. Vestibulum posuere risus sed ante interdum tempor. Sed nec enim lorem, eu ultrices nisl. Maecenas cursus lacus non metus convallis cursus. Cras egestas feugiat libero, quis luctus justo mattis ut. Proin vitae magna et purus convallis blandit ut sed risus. Nulla vel ligula massa, vel interdum tortor. Integer sit amet eros sit amet lectus suscipit adipiscing. Aenean sagittis diam eu orci porttitor luctus.

Sed sed elit eu ante malesuada aliquet. Nunc quis quam turpis, accumsan faucibus sapien. Nam eget sem velit, id eleifend elit. Phasellus auctor egestas libero a adipiscing. Integer convallis, nunc sed tincidunt sollicitudin, augue diam accumsan risus, sit amet lacinia neque quam at lorem. Quisque ac rhoncus mauris. Donec felis odio, consequat vitae pretium ullamcorper, blandit id lorem. Morbi nec tellus vitae velit eleifend bibendum at vel eros. Sed quis enim urna. Cras placerat ultricies ipsum, sed tempor lorem placerat vitae. Morbi tellus leo, blandit eget feugiat nec, bibendum egestas erat. Mauris lobortis congue vestibulum. Nunc ultricies arcu in neque hendrerit convallis. Phasellus facilisis dictum nulla vel iaculis. Donec in adipiscing tellus. Aliquam non turpis neque. Nam in velit nec tortor gravida convallis. Vivamus tincidunt eros quis elit ultricies at adipiscing urna tempor. Donec posuere, ante pulvinar laoreet adipiscing, eros nulla pharetra nulla, vel rutrum elit est vitae ligula.

Phasellus urna nulla, laoreet in imperdiet a, porttitor eu nibh. Phasellus ac arcu magna, sit amet dapibus eros. Maecenas eget metus vitae leo feugiat lobortis ac quis tortor. Pellentesque habitant morbi tristique senectus et netus et malesuada fames ac turpis egestas. Pellentesque vel augue metus. Quisque fringilla, arcu ut fringilla consectetur, mauris tellus euismod sem, in vehicula est leo vel nisi. Praesent sapien ligula, malesuada eget hendrerit ac, aliquet nec turpis. Cum sociis natoque penatibus et magnis dis parturient montes, nascetur ridiculus mus. Nulla facilisi. Mauris commodo elementum elit, fermentum placerat nulla iaculis nec. Vestibulum neque dui, cursus at pharetra eu, sollicitudin eget justo. In id dignissim augue. Fusce interdum malesuada ipsum, non aliquet quam auctor eget. Curabitur turpis libero, pharetra ut cursus porta, pellentesque vitae magna. Cras porttitor interdum nisl \cite{raymond1999}.


%%
%% Parte pos-textual
%%
\backmatter

% Apendices
% Comente se n??o houver ap??ndices
\appendix

% Eh aconselhavel criar cada apendice em um arquivo separado, digamos
% "apendice1.tex", "apendice.tex", ... "apendiceM.tex" e depois
% inclui--los com:
% \include{apendice1}
% \include{apendice2}
% ...
% \include{apendiceM}


% Bibliografia
% ?? aconselh??vel utilizar o BibTeX a partir de um arquivo, digamos "biblio.bib".
% Para ajuda na cria????o do arquivo .bib e utiliza????o do BibTeX, recorra ao
% BibTeXpress em www.cin.ufpe.br/~paguso/bibtexpress
\bibliographystyle{abntex2-alf}
\bibliography{biblio}

% Colofon
% Inclui uma pequena nota com referencia a UFPEThesis
% Comente para omitir
%\colophon

%% Fim do documento
\end{document}
