\documentclass[msc]{mestrado}

\usepackage{master}

%% Inicio do documento
\begin{document}
\mainmatter

\headsep=40pt
\oddsidemargin=15pt
\begin{spacing}{1.4}
\parskip=6pt

%\chapter{Introdu��o}

%\chapter{Ethernet e Tempo Real}
\label{cap:motivacao}

%\chapter{O Protocolo \doris{}} % (10 pag)
\label{cap:doris}

%\chapter{A disciplina \emph{DoRiS}} % (10 pag)
\label{cap:specVerif}

%\chapter{Plataforma Operacional}
\label{cap:platOp}

\xchapter{Especificação do problema e a solução proposta}{Opcional}

Detalhes do trabalho a resolver.

Outro parágrafo.

\section{Sistemas de tempo real}
\section{Raspberry}


\end{spacing}

%%
%% Parte p�s-textual
%%
\backmatter

% Bibliografia
% � aconselh�vel utilizar o BibTeX a partir de um arquivo, digamos "biblio.bib".
% Para ajuda na cria��o do arquivo .bib e utiliza��o do BibTeX, recorra ao
% BibTeXpress em www.cin.ufpe.br/~paguso/bibtexpress
\renewcommand{\bibname}{REFER�NCIAS}
%\nocite{*}
%\bibliographystyle{authordate2}
%\bibliographystyle{abnt-alf}

\bibliography{../../bib}

%% Fim do documento
\end{document}
