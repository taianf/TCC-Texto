\documentclass[msc]{mestrado}

\usepackage{master}

%% Inicio do documento
\begin{document}
\mainmatter

\headsep=40pt
\oddsidemargin=15pt
\begin{spacing}{1.4}
\parskip=6pt


%\chapter{Introdu��o}
\label{cap:introducao}

%\chapter{Ethernet e Tempo Real}
%\label{cap:motivacao}
\chapter{Ethernet e Tempo Real}
\label{cap:motivacao}

A se��o \ref{sec:caracEther} descreve as caracter�sticas f�sicas do padr�o Ethernet
assim como o protocolo de acesso ao meio CSMA/CD \cite{CSMA/CD01}.  Algumas das
solu��es propostas para tornar Ethernet determinista ser�o apresentadas na se��o
\ref{sec:etherDeterm}.  Duas destas propostas ser�o detalhadas na se��o
\ref{sec:TempraVTPE}, pois serviram de fontes de inspira��o a este
trabalho. Finalmente, considera��es sobre sistemas h�bridos ser�o discutida na se��o
\ref{sec:sisHibrid}.

\section{Ethernet}
\label{sec:caracEther}

\subsection{Caracter�sticas f�sicas}
\label{sec:carFisis}

O padr�o de comunica��o Ethernet compartilhada (modalidade \emph{half-duplex})
\cite{CSMA/CD01} define um barramento Ethernet como um conjunto de esta��es
utilizando um mesmo barramento (meio f�sico) e trocando mensagens entre elas.  Um
barramento Ethernet caracteriza um \textbf{dom�nio de colis�o}, isto porque as
mensagens emitidas por duas esta��es do barramento podem colidir de acordo com o
protocolo CSMA/CD, como ser� visto na se��o \ref{sec:CSMACD}.
 
As mensagens, encapsuladas de acordo com o padr�o (ver figura
\ref{fig:ethernetFrame}), s�o chamadas de quadro Ethernet, ou simplesmente
quadro. Durante a transmiss�o de um quadro, diz-se que o barramento est�
\textbf{ocupado}, enquanto que, na aus�ncia de transmiss�o, o barramento � dito
\textbf{livre}.  No fim da transmiss�o de um quadro, a transi��o de estado do
barramento provoca uma interrup��o em todas as esta��es, chamada de interrup��o de
fim de quadro (\emph{End-Of-Frame interrupt}) e denotada EOF.

Dois tempos caracterizam a transmiss�o de um quadro no barramento: o
tempo de transmiss�o e o tempo de propaga��o.  O tempo de propaga��o
$T_{prop}$ de um pacote de uma extremidade a outra do barramento
Ethernet s� depende das caracter�sticas do meio f�sico, isto �, da
velocidade de propaga��o de um sinal el�trico na rede e do comprimento
total do barramento.  Nas redes 10Mbps e 100Mbps, o valor m�ximo deste
comprimento � $500 m$. Considerando a velocidade de propaga��o
constante de $2,5\,10^{8}\,m/s$, tem-se $T_{prop} \approx 2\,\mu s$.

Para caracterizar o tempo de transmiss�o de maneira independente da largura de
banda, utiliza-se a unidade temporal elementar \ing{bit-time} (bT) definido da
seguinte maneira: $1 bT$ � o tempo que um bit leva para ser transmitido no meio
f�sico.  Por exemplo, numa rede \emph{Fast}-Ethernet (100Mbps), $1\,bT = 10\,ns$.
J� numa rede Gigabits (1Gbps), este valor � $1\,ns$.

Entre duas transmiss�es consecutivas, � preciso reservar um tempo de recupera��o
para que as diferentes camadas da pilha de rede possam esvaziar as suas mem�rias
locais de recep��o e redefinir as vari�veis do protocolo CSMA/CD.  Portanto, a
transmiss�o de dois quadros consecutivos deve sempre ser separada por um tempo
m�nimo, chamado \emph{\textbf{Interframe Gap} (IFG)}, durante o qual o meio
permanece livre.  Nas redes 10 e 100Mbps, $IFG = 96\,bT$.

\begin{figure}[tb]
  \centering
  \input{fig/ethernetFrame.pstex_t}  
  \caption{Um quadro Ethernet (n�meros em \ing{bytes})}
  \label{fig:ethernetFrame} 
\end{figure}

A figura \ref{fig:ethernetFrame} apresenta o detalhe do formato de um quadro
Ethernet com a indica��o do tamanho dos campos em bytes.  Os \ing{bytes} do
pre�mbulo e \textit{Start of frame} (SOF) marcam o in�cio de um quadro. Eles s�o
associados � camada f�sica e servem para sincronizar o rel�gio da interface com a
freq��ncia do sinal de entrada.  Em seguida, o quadro cont�m o cabe�alho Ethernet,
com os campos de endere�os da esta��o  de destino e da esta��o de origem, e o campo
\emph{type field} utilizado para definir o tipo ou o tamanho do quadro. Os demais
cabe�alhos associados a protocolos de rede tais como IP ou TCP (n�o mostrados na
figura) s�o encapsulados juntos com os dados de aplica��es.  Depois deste segmento de
dados de tamanho variando entre 46 e 1500 \ing{bytes}, o campo \emph{Frame Check
  Sequence} (FCS) termina o quadro. Este campo serve para conferir a integridade do
quadro depois da sua recep��o.

Deduz-se que no padr�o Ethernet, o tamanho dos quadros varia de 64 a 1518 bytes. Um
quadro de tamanho m�nimo de 64 bytes (512 bits) � chamado de \ing{slot time}.
Usualmente, n�o se considera no tamanho de um quadro os campos do pre�mbulo e do
SOF, pois s�o associados � camada f�sica.  No entanto, o c�lculo do tempo de
transmiss�o de um quadro deve levar em conta estes 8 bytes iniciais.

Denota-se $\boldsymbol{\delta}$ o intervalo de tempo necess�rio para transmitir um
quadro m�nimo de tamanho $576$ bits (7 bytes). Na taxa de 100Mbps, $\delta = 5,76\,\mu
s$.  Denota-se $\delta_m$ o tempo m�ximo de transmiss�o de um quadro de 1526
bytes. Na taxa de 100Mbps, $\delta_m = 12.208\,bT\,=\,122,08\,\mu s$.

\subsection{Controle de acesso ao meio}
\label{sec:CSMACD}

Como foi visto na se��o anterior, o barramento Ethernet constitui um recurso
compartilhado pelas esta��es de um mesmo barramento.  Para permitir o desempenho da
comunica��o e o sucesso das transmiss�es, � preciso definir uma pol�tica justa,
eficiente e confi�vel que organiza o acesso ao meio para que as esta��es possam se
comunicar.  Podemos expressar estas propriedades pelos seguintes requisitos:
\begin{enumerate}
\item Uma esta��o que quer transmitir conseguir� o meio (\emph{liveness});
\item Um quadro transmitido chega sem altera��o ao seu destinat�rio
  (\emph{safety});
\item Todas as esta��es t�m o mesmo direito de acesso ao meio
  (\emph{fairness}).
\end{enumerate}

Para atender a estes requisitos, o mecanismo de Controle do Acesso ao Meio (MAC)
utiliza a capacidade que as esta��es t�m de monitorar o meio f�sico para detectar se
seu estado est� livre ou ocupado \cite{CSMA/CD01,Wang99,Wang02}.  De forma resumida,
o protocolo funciona da seguinte maneira.  Todas as esta��es monitoram o meio de
maneira cont�nua.  Quando uma esta��o quer transmitir, ela espera detectar o meio
livre durante um tempo padr�o igual ao \emph{Interframe Gap} (IFG).  Depois deste
tempo, ela come�a a emitir imediatamente.  Enquanto ela est� transmitindo, a esta��o
continua monitorando o meio durante um \ing{slot time}.  Dois cen�rios podem ent�o
acontecer.  No primeiro, a esta��o consegue transmitir durante um \ing{slot time}
sem perceber nenhuma diferen�a entre o sinal que ela transmite e o sinal que ela
monitora. Este � o cen�rio de transmiss�o com sucesso.  No segundo, a esta��o
detecta uma altera��o entre o sinal que ela est� transmitindo e o sinal que ela est�
recebendo. Se isto ocorre, uma colis�o est� acontecendo, ou seja, uma outra esta��o
come�ou a emitir um quadro simultaneamente.  Neste caso, a esta��o p�ra de
transmitir o seu quadro e transmite uma seq��ncia padr�o de 48 bits, chamada de
\ing{jam}, para garantir que todas as esta��es do barramento detectam a colis�o.
Depois, ela, e as demais esta��es que participaram da colis�o, entram em estado de
\ing{backoff} antes de tentar transmitir novamente.

O tempo de \emph{backoff} � um m�ltiplo do \emph{slot time}.  Um fator
multiplicativo inteiro $K$ � escolhido aleatoriamente dentro de um intervalo
exponencialmente crescente de acordo com o n�mero de colis�es. Depois de $n$
colis�es, $K$ � escolhido no conjunto $\{0, 1, 2, \ldots, 2^{m-1}\}$ onde $m
= min(n,10)$.  Depois da en�sima colis�o, a esta��o espera ent�o $K.512 \, bT$ antes
de poder tentar transmitir novamente. Desta forma, a probabilidade que aconte�am
colis�es em s�rie decresce exponencialmente, tal como sugere o nome \emph{Binary
  Exponencial Backoff} (BEB) deste algoritmo, que � parte do protocolo CSMA/CD.

Num barramento 100Mbps de $500 m$ de comprimento, o tamanho m�nimo de
$72$ bytes para um quadro Ethernet garante a detec��o de uma colis�o
no pior caso. Para entender esta propriedade, imagine o seguinte
cen�rio envolvendo duas esta��es $A$ e B a uma dist�ncia de $500 m$
uma da outra. O tempo de propaga��o de $A$ a $B$ � aproximadamente
$T_{prop} = 2 \mu s$ (ver se��o \ref{sec:carFisis}).  Suponha que o
meio est� inicialmente livre. Num instante $t_1$, a esta��o $A$ come�a
a emitir um quadro $q$ de tamanho m�nimo.  Num instante $t_2$, logo
antes da chegada de $q$, a esta��o $B$ come�a a emitir. Em seguida,
uma colis�o ocorre num instante $t_c$ tal que $t_2 \leqslant t_c
\leqslant t_1 + T_{prop} $.  Observe que no instante $t_c$, $A$ ainda
n�o terminou de enviar $q$, cuja transmiss�o leva $\delta = 5,76 \mu
s$. No instante $t_1 + T_{prop}$, $B$ percebe a colis�o, e
conseq�entemente, p�ra de transmitir seu quadro e come�a a emitir um
quadro \ing{jam}. Este quadro se propaga no meio e chega em $A$ num
instante $t_3$ tal que $t_3 \leqslant t_1 + 2 T_{prop}$.  A condi��o
$2 T_{prop} < \delta$ garante portanto que o quadro \ing{jam} chegue
em $A$ antes que $A$ tenha terminada a transmiss�o do quadro $q$
envolvido na colis�o.  Desta forma, $A$ percebe a colis�o envolvendo
$q$. Al�m disto, o quadro \ing{jam} garante que a percep��o da colis�o
acontece tanto em $A$ quanto em todas as outras esta��es conectadas ao
barramento.

Num outro cen�rio, considerando, por exemplo, duas esta��es distante
de $1.000m$, o tempo de propaga��o entre $A$ e $B$ seria ent�o de $4
\mu s$. Neste caso, a informa��o da colis�o poder� chegar em $A$ quase
$8 \mu s$ depois do in�cio da transmiss�o de $q$.  Depois de tanto
tempo, n�o somente a transmiss�o de $q$ j� poder� ter sido terminada,
como tamb�m, $A$ poder� j� ter iniciado uma nova transmiss�o. O quadro
\ing{jam} ser� ent�o associado erradamente ao segundo quadro.
Percebe-se, portanto, que a condi��o $2 T_{prop} < \delta$ �
necess�ria para que o mecanismo de detec��o das colis�es funcione
corretamente.


\subsection{Probabilidade de colis�o}
\label{sec:probCol}

Considerando duas esta��es $A$ e $B$ isoladas, a probabilidade m�xima
de colis�o entre estas duas esta��es acontece quando $A$ e $B$ est�o
distantes o m�ximo poss�vel uma da outra, isto �, de 500m. Imagine que
$A$ come�a a transmitir, a probabilidade que uma colis�o ocorra � a
probabilidade que $B$ come�a a transmitir antes de ter percebido que
$A$ j� est� transmitindo, ou seja, antes de $T_{prop}$. Esta
probabilidade depende da freq��ncia de transmiss�o de quadros por B.
Mas, de qualquer forma, ela � relativamente pequena, pois $T_{prop}$ �
pequeno em compara��o ao tempo m�dio de transmiss�o de quadros.

No entanto, um efeito de sincroniza��o das esta��es pode acontecer e
aumentar a probabilidade de colis�o significativamente.  Efetivamente,
quando duas esta��es esperam o meio ocupado por uma terceira, elas
sofrem um efeito de sincroniza��o provocado pela espera conjunta do
meio.  A figura \ref{fig:colisaoCSMACD} ilustra um cen�rio de colis�o,
provocada pela espera conjunta com tr�s esta��es $A$, $B$ e $C$
competindo pelo meio. Nesta figura, os tamanhos dos intervalos de
tempo s�o apenas ilustrativos.

No cen�rio de ``espera conjunta'', suponha que a esta��o $A$ esteja entre as duas
esta��es $B$ e $C$ e aproximadamente a igual dist�ncia de ambas.  Quando as esta��es $B$ e
C tentam transmitir, elas constatam que o meio est� ocupado pela esta��o $A$ que j�
est� transmitindo. Portanto, elas esperam at� sentir o meio livre. Quando a
interrup��o de EOF provocada pelo fim da transmiss�o de $A$ acontece, $B$ e $C$ esperam
$IFG$ antes de come�ar a transmitir. Como as dist�ncias entre $A$ e $B$ e entre $A$ e C
s�o quase iguais, a interrup��o EOF do fim da transmiss�o de $A$ chega em $B$ e C
aproximadamente no mesmo instante. Portanto, depois de $IFG$, ambas come�am a
transmitir simultaneamente, resultando numa colis�o, pois, ap�s um tempo curto,
as duas esta��es observam as diferen�as entre os sinais que elas est�o emitindo e
recebendo. Conseq�entemente, ap�s diagnosticar a colis�o, ambas mandam imediatamente
as mensagens \emph{jam} antes de parar de transmitir e entrar em estado de
\emph{backoff} por um tempo aleat�rio, assim como foi visto no in�cio desta
se��o. Se os tempos de \emph{backoff} escolhidos foram diferentes, as duas esta��es
conseguem transmitir com sucesso. Sen�o, uma nova colis�o acontece e as
esta��es entram novamente em estado de \emph{backoff}.

\begin{figure}[bt]
  \index{figuras!colisaoCSMACD}%
  \centering \input{fig/colisaoCSMA-CD.pstex_t}
  \caption{Um cen�rio de colis�o \label{fig:colisaoCSMACD}}
\end{figure}

Este cen�rio simples mostra o principal mecanismo respons�vel pelas colis�es na
Ethernet e, portanto, o car�ter n�o determinista do algoritmo BEB do protocolo
CSMA/CD.  Como os tempos de \emph{backoff} s�o aleat�rios, eles causam atrasos n�o
deterministas nas entregas dos quadros. Utilizando um modelo probabil�stico, a
distribui��o dos atrasos em fun��o da carga da rede pode ser estimada teoricamente
\cite{Schneider00,Laqua02}.

O cen�rio do pior caso � chamado de ``efeito de captura''.  Continuando o cen�rio da
figura \ref{fig:colisaoCSMACD}, o efeito de captura do meio acontece, por exemplo,
quando ocorrem colis�es sucessivas entre $C$ e uma ou mais esta��es. No exemplo da
figura \ref{fig:colisaoCSMACD}, quando $B$ termina de transmitir, $C$ come�a a
transmitir logo em seguida. Suponha que uma outra esta��o, eventualmente B, come�a a
transmitir tamb�m, provocando uma nova colis�o do quadro de C. O contador de
colis�es de $C$ � incrementado, e portanto, o intervalo de escolha do n�mero
aleat�rio de \ing{backoff} � multiplicado por 2.  Logo, a probabilidade que $C$
ganhe o acesso ao meio diminui. A cada nova colis�o do quadro de $C$ com um quadro
ainda n�o envolvido em colis�o alguma, a probabilidade que $C$ ganhe o acesso ao
meio � dividida por 2.  No pior caso, este efeito provoca o descarte daquele quadro,
depois de 16 tentativas \cite{Wang02,Decotignie05}. S� ent�o, com o custo do
descarte de um quadro, a esta��o volta a competir para o meio com o seu contador de
colis�es igual a 0, e conseq�entemente, com a maior probabilidade de ganhar o meio
em casos de colis�o. A possibilidade de atrasos ou mesmo de perdas de quadros do
algoritmo BEB do protocolo CSMA/CD torna este protocolo impr�prio para ambientes
\emph{hard real time} \cite{Wang99}.

\section{Ethernet: o desafio do determinismo}
\label{sec:etherDeterm}

As propostas para aumentar a previsibilidade das redes Ethernet e oferecer garantias
temporais �s aplica��es t�picas de plantas industriais encontram-se em grande n�mero
na literatura \cite{Hanssen03, Decotignie05}.  Na modalidade Ethernet compartilhada
(\ing{half-duplex}), distinguem-se duas classes principais de solu��es. Aquelas
baseadas em modifica��es do hardware dos cart�es Ethernet s�o apresentadas
na se��o \ref{sec:ethHardware} e as outras baseadas em software ser�o descritas na
se��o \ref{sec:ethSoftware}. Na se��o \ref{sec:ethSwitch}, a modalidade Ethernet
comutada (\ing{full-duplex}) ser� discutida, pois � a modalidade dominante no
mercado.

\subsection{Solu��es baseadas em modifica��es do hardware padr�o}
\label{sec:ethHardware}

As solu��es baseadas em hardware modificam a camada MAC do protocolo CSMA/CD para
diminuir ou mesmo anular a probabilidade de perda de quadro. Esta se��o apresenta
brevemente as principais solu��es que seguem esta abordagem.

O protocolo CSMA/CA (\ing{Collision Avoidance}), usado no padr�o 802.11 para redes
Ethernet sem fio \cite{Crow97, 80211}, implementa um tempo de espera aleat�rio
(\ing{backoff}) n�o somente depois de uma colis�o (como o CSMA/CD), mas tamb�m
quando uma esta��o est� esperando para o meio ficar livre, na situa��o de ``espera
conjunta'' descrita na se��o \ref{sec:probCol}. Suponha que uma
esta��o queira transmitir uma mensagem, mas que o meio esteja ocupado. Ela espera
at� sentir o meio livre, mas ao contr�rio do CSMA/CD, quando ocorre o EOF, a esta��o
n�o transmite imediatamente. Ela entra em estado de \ing{backoff} para um tempo de
espera aleat�rio antes de tentar emitir. Desta forma, o efeito de sincroniza��o pela
``espera conjunta'' � diminu�do \cite{Kurose05}.  No caso de uma colis�o, o tamanho
do dom�nio de escolha do tempo de espera aleat�rio aumenta exponencialmente, assim
como para o CSMA/CD.

O protocolo CSMA/DCR (\ing{Deterministic Collision Resolution}) \cite{LeLann93}
utiliza estruturas de dados adequadas para eliminar esta��es da competi��o para o
meio depois de uma colis�o. Resumidamente, o conjunto de esta��es de um segmento Ethernet
� dividido em uma �rvore bin�ria de sub-conjuntos. A cada colis�o sucessiva que
ocorre, um dos dois sub-conjuntos ainda participando da competi��o pelo meio �
retirado da competi��o.  No caso do protocolo, CSMA/BLAM (\ing{Binary Logarithmic
  Arbitration Method}) \cite{Molle94}, uma fase de arbitragem � utilizada depois de
uma colis�o para garantir que todas as esta��es competindo pelo meio utilizam o mesmo
contador de colis�o. Desta forma, todas as esta��es em competi��o t�m chances
iguais de ganhar o meio.  Apesar de serem deterministas, estes dois protocolos
apresentam uma variabilidade significativa nos tempos de respostas, entre o pior e o
melhor caso. Esta variabilidade � indesej�vel em sistemas de tempo real,
principalmente para aqueles que t�m caracter�sticas peri�dicas (sistemas de controle,
por exemplo).

Outras solu��es utilizam um tamanho vari�vel do quadro \ing{jam}. Exemplos desta
abordagem s�o os protocolos CSMA/PRI \cite{Guo05} \ing{Priority Reservation by
Interruptions} e o EQuB \cite{Sobrinho98}. Estes protocolos permitem definir
prioridades entre as esta��es da seguinte forma.  Quando uma esta��o quer ter acesso
ao meio, apesar de este estar ocupado, ela come�a a transmitir sem esperar que o meio
fique livre.  Desta forma, ela provoca uma colis�o. Transmitindo uma mensagem
\ing{jam} de tamanho predefinido, ela informa �s demais esta��es que elas devem
abandonar a competi��o pelo meio. As prioridades das esta��es s�o associadas
ao tamanho dos quadros \ing{jam}, sendo o maior tamanho associado a esta��o
com a prioridade mais alta.

Uma outra abordagem com prioridades utiliza dois valores de $IFG$ diferentes para as
comunica��es com ou sem requisitos temporais \cite{LoBello01}.  A prioridade alta
corresponde ao valor de $IFG$ definido pela norma CSMA/CD \cite{CSMA/CD01}. A
prioridade baixa utiliza este valor de $IFG$ aumentado de 512 bT. Para enviar uma
mensagem com prioridade baixa, um processo deve observar o meio livre durante este
tempo maior. Portanto, qualquer mensagem com prioridade alta ser� transmitida
antes. Esta solu��o permite o isolamento efetivo dos dois tipos de comunica��es mas
n�o elimina a possibilidade de colis�o entre mensagens de alta prioridade.

� importante observar que as solu��es baseadas em prioridades n�o garantem
determinismo, a n�o ser quando a camada de controle l�gico (LLC) utiliza alguma
pol�tica de escalonamento dos envios.

Apesar das suas qualidades, as solu��es com modifica��o do hardware n�o permitem o
aproveitamento dos dispositivos existentes que s�o de baixo custo e de grande
disponibilidade no mercado. Al�m disso, as altera��es de hardware comprometem a
compatibilidade do protocolo com o padr�o Ethernet 802.3.  Neste trabalho, uma
abordagem mais flex�vel foi escolhida, onde o determinismo da rede � garantido
exclusivamente no n�vel de software.


\subsection{Solu��es baseadas em software}
\label{sec:ethSoftware}

As solu��es baseadas em software consistem em estender a camada MAC do protocolo
CSMA/CD, utilizando a sub-camada de controle l�gico (LLC) localizada
entre a camada de rede e a camada MAC. Esta camada intermedi�ria filtra as mensagens
de acordo com regras predefinidas de forma a evitar as colis�es ou reduzir a
probabilidade de elas acontecerem.

Um dos mecanismos mais simples para oferecer garantias temporais � a divis�o do meio
f�sico em ciclos temporais com atribui��o de \emph{slots} temporais de emiss�o para
cada esta��o. O exemplo mais conhecido desta id�ia � o \emph{Time Division Multiple
  Access} (TDMA) \cite{Kurose05}. Apesar de prover alto grau de determinismo, o TDMA
apresenta um problema de desempenho em situa��o de baixa carga da rede, j� que mesmo
que uma esta��o n�o tenha nada para transmitir, nenhuma outra esta��o pode
aproveitar o \emph{slot} de transmiss�o dispon�vel.

Uma outra abordagem � a utiliza��o do modelo Mestre-Escravo. Uma esta��o (o mestre) 
gerencia a rede e distribui as autoriza��es de emiss�o por meio de mensagens para as
demais esta��es (os escravos).  Baseado neste modelo, o protocolo FTT-Ethernet
\cite{Pedreiras02} permite a coexist�ncia de comunica��o cr�tica e n�o-cr�tica.
Nesta arquitetura, o mestre constitui um ponto �nico de falha, que deve ser
replicado a fim de prover toler�ncia a falhas.  Devido a esta estrutura
centralizada, a assimetria nas cargas da rede podem implicar problemas de extens�o
e de desempenho do protocolo.

V�rios protocolos, tais como 802.5 e FDDI \cite{Hanssen03}, RETHER
\cite{Venkatrami94} e RTnet \cite{Hanssen05} utilizam arquiteturas baseadas em anel
l�gico e bast�o circulante (\emph{token}) expl�cito para carregar as reservas e os
direitos de acesso ao meio. Quando uma esta��o recebe o bast�o circulante, ela
adquire o direito de transmitir. Depois de completar a transmiss�o de uma ou mais
mensagens, ela transmite o bast�o circulante para o seu sucessor no anel l�gico.  O
bast�o circula regularmente, passando em todas as esta��es do anel uma
ap�s a outra. Algumas destas solu��es utilizam o bast�o circulante para transmitir
informa��es de prioridades e de tempos alocados.  No caso geral, estas solu��es
conseguem oferecer garantias temporais. No entanto, elas apresentam dois problemas.
Primeiro, o tempo de transmiss�o do bast�o circulante gera uma sobrecarga que pode
ser significativa quando a maioria das mensagens s�o de tamanho m�nimo. Segundo, no
caso da perda do bast�o circulante, o tempo de recupera��o, necess�rio para detectar
a falha e criar um novo bast�o circulante, � maior do que o tempo de transmiss�o do
bast�o circulante na aus�ncia de falhas. Isto causa um indeterminismo potencial na
entrega das mensagens, pois as falhas ocorrem de forma n�o previs�vel.

Para tentar resolver as limita��es dos protocolos com bast�o circulante expl�cito, novas
abordagens foram desenvolvidas com mecanismos impl�citos de passagem do
bast�o circulante. Muitas vezes, estas abordagens utilizam temporizadores
\cite{Lamport84} para definir bifurca��es no fluxo de execu��o de um processo, em
fun��o de condi��es temporais e l�gicas. Observando a comunica��o, cada esta��o
determina o instante no qual ela tem direito de transmitir em fun��o da sua posi��o
no anel l�gico e do valor do seu temporizador.  Dois protocolos utilizando esta
abordagem com bast�o circulante impl�cito ser�o descritos detalhadamente na se��o
\ref{sec:TempraVTPE}.

Outros protocolos utilizam mecanismos baseados em janelas temporais ou em tempo
virtual \cite{Hanssen03}, mas estas abordagens s�o probabil�sticas e n�o oferecem
garantias temporais para sistemas cr�ticos. Tamb�m probabil�sticas, mas com
mecanismos diferentes, as t�cnicas de \emph{smoothing} \cite{Carpenzano02} consistem
em observar os padr�es de comunica��o na rede para impedir que haja transmiss�es em
rajadas (\emph{bursts}) \cite{Decotignie05}.


\subsection{Ethernet comutada}
\label{sec:ethSwitch}

A tecnologia Ethernet comutada vem sendo muito utilizada nos �ltimos
anos. Nesta arquitetura de rede, todas as esta��es s�o conectadas a um comutador
que disp�e de mem�rias de recep��o e emiss�o para armazenar e transmitir as
mensagens.  Quando uma esta��o $A$ quer emitir uma mensagem $m$ para uma esta��o B,
ela a envia para o comutador o qual a encaminha para B. Se o comutador j�
estiver transmitindo uma mensagem para $B$ quando $m$ chega, ele coloca $m$ na fila
das mensagens esperando para serem transmitidas para B.  Desta forma, a utiliza��o
de um comutador elimina o compartilhamento do meio f�sico entre as esta��es e a
exist�ncia das colis�es decorrentes.  Al�m disso, nesta tecnologia, uma esta��o pode
transmitir e receber ao mesmo tempo, multiplicando por dois a vaz�o total dispon�vel
para a comunica��o.

Apesar destas vantagens, a tecnologia Ethernet comutada apresenta v�rios
desafios relacionados a sua utiliza��o em redes de controle industriais.  Em particular, a
utiliza��o de filas no comutador dificulta a implementa��o de comunica��o
\emph{broadcast} r�pida e determinista por duas raz�es. Em primeiro lugar, o
processamento das mensagens (recep��o, roteamento e transmiss�o) no comutador tem
um custo temporal de v�rios $\mu s$ \cite{Wang02}, isso mesmo na aus�ncia de filas
de espera. Em segundo lugar, o comutador constitui um ponto de gargalo na
comunica��o, como ilustra o seguinte cen�rio. Suponha que v�rias esta��es mandem
mensagens para o mesmo destinat�rio $A$ de tal forma que a taxa acumulada de chegada
de mensagem para $A$ no comutador seja v�rias vezes maior que a taxa de
transmiss�o do canal conectando o comutador a A.  Portanto, a fila de espera, que tem
um tamanho limitado, acaba se enchendo at� provocar perdas de mensagens.

Tanto em Ethernet comutada quanto em Ethernet compartilhada, a abordagem do
protocolo ``Time Triggered Ethernet'' resolve estas limita��es \cite{Kopetz05},
oferecendo previsibilidade temporal e seguran�a no funcionamento. No entanto, estas
garantias dependem de uma an�lise de escalonamento em tempo de projeto, e portanto,
esta abordagem n�o permite reconfigura��o da rede em tempo de execu��o.


\section{Os protocolos TEMPRA e VTPE}
\label{sec:TempraVTPE}

Como foi visto, nem todas as abordagens descritas conseguem atender aos requisitos
espec�ficos dos sistemas de tempo real cr�ticos. Algumas apresentam pontos de falhas
�nicos, ou tempo de lat�ncia significativo na ocorr�ncia de certos eventos (ex:
perda do bast�o circulante). As solu��es baseadas em Ethernet comutada s�o
bastante eficientes, mas dificultam a implementa��o de comunica��o ``um-para-muitos''
deterministas. Al�m disso, perdas de mensagens podem ocorrer nos comutadores em
caso de congestionamento nas mem�rias internas \cite{Decotignie05}.

No contexto de um barramento Ethernet compartilhada, o protocolo VTPE \ing{Virtual
  Token Passing Ethernet} desenvolvido por Carreiro et al \cite{Carreiro03}, combina
as abordagens de \emph{token impl�cito} e TPR (\emph{Timed Packet Release})
encontrada no protocolo TEMPRA \cite{Pritty95}.  Estes dois protocolos, VTPE e
TEMPRA, constitu�ram as duas fontes principais de inspira��o do nosso
trabalho. Portanto, dedicaremos a pr�xima se��o �s suas descri��es detalhadas.

\subsection{Modelo \ing{publish-subscribe}}
\label{sec:modelo}

Os dois protocolos apresentados nesta se��o utilizam o modelo de comu\-ni\-ca\-��o
\ing{pub\-lish-sub\-scribe} \cite{Dolejs04}.  Quando uma esta��o precisa emitir uma
mensagem, ela a publica no meio f�sico, utilizando no campo de destino o endere�o
para a comunica��o um-para-todos, reservado pelo padr�o Ethernet (48 bits com valor
1).  Portanto, qualquer mensagem transmitida � recebida por todas as esta��es do
barramento. Na sua recep��o, uma esta��o determina quem publicou a mensagem atrav�s
do campo ``endere�o de origem'' do quadro Ethernet (ver figura
\ref{fig:ethernetFrame}). Ela ent�o decide o que fazer com a mensagem: descart�-la
ou encaminh�-la para a camada superior.

Como pode ser notado, este modelo de comunica��o � coerente com o uso de Ethernet
compartilhada, j� que esta tecnologia permite disponibilizar modos de comunica��o
ponto-a-ponto, um-para-muitos e um-para-todos de forma eficiente. Por outro lado, o
modelo \ing{publish-subscribe} implica que todos os membros de um mesmo barramento
recebem todas as mensagens publicadas. Isto tem conseq��ncias no que se refere ao
uso da banda, como ser� descrito na se��o \ref{sec:sisHibrid}.

No TEMPRA e no VTPE, assim como no protocolo de bast�o circulante \cite{TokenBus},
as esta��es s�o organizadas em um anel l�gico. Cada esta��o tem um identificador
inteiro �nico, variando de $1$ a $N$, onde $N$ � o n�mero de esta��es do
barramento. Define-se o tempo de rota��o do bast�o circulante ($T_{RT}$) como sendo
o tempo entre duas passagens consecutivas do bast�o circulante numa mesma esta��o.

No caso do protocolo TEMPRA, assume-se que o anel l�gico coincide com o anel f�sico,
ou seja, quanto maior � o identificador da esta��o, maior � a dist�ncia desta
esta��o � primeira esta��o da rede. No caso do VTPE, a ordem dos identificadores �
arbitr�ria. A figura \ref{fig:aneis} apresenta estes an�is para estes dois
protocolos. As letras correspondem �s esta��es e o eixo horizontal representa as
dist�ncias entre elas. Os n�meros correspondem aos identificadores e as linhas
pontilhadas indicam o anel l�gico associado.

\begin{figure}[bt]
  \index{figuras!aneis}%
  \centering
{\scalebox{1}{
  \centering \input{fig/aneis.pstex_t}
 }}
  \caption{Os an�is VTPE e TEMPRA \label{fig:aneis}}
\end{figure}

O protocolo TEMPRA necessita da adi��o de um hardware espec�fico, enquanto o
protocolo VTPE n�o envolve modifica��es da camada MAC.


\subsection{O protocolo TEMPRA}
\label{sec:TEMPRA}

O protocolo TEMPRA \cite{Pritty95} utiliza o mecanismo de \emph{Timed Packet
  Release} (TPR) para impedir as colis�es inerentes ao protocolo CSMA/CD. Quando uma
esta��o precisa de um servi�o com requisitos temporais cr�ticos, ela emite uma
requisi��o e as demais esta��es do barramento mudam do protocolo CSMA/CD para o
protocolo TEMPRA, utilizando uma outra interface de rede dedicada � comunica��o de
tempo real. Depois que a comunica��o de tempo real termina, as esta��es voltam para
o CSMA/CD.

Para organizar o servi�o de tempo real, assume-se que a primeira esta��o (que tem o
menor identificador do anel) est� posicionada numa extremidade do barramento.  Esta
esta��o, chamada de \emph{monitor}, � selecionada para emitir um pulso (\ing{slot
  pulse}), com um per�odo predefinido.  Este sinal, de dura��o $t_s$, tem um padr�o
�nico e se propaga unidirecionalmente no barramento, dando uma refer�ncia temporal
sem ambig�idade para as esta��es, atrav�s da sua interrup��o de fim de quadro (EOF).
Na passagem de um pulso, cada esta��o inicia um temporizador.  Estes seguem uma
regra aritm�tica de par�metro $t_d = 1 \, \mu s$, ou seja, o temporizador da esta��o
de identificador $id$ vale $id * t_d$.  Para adquirir o bast�o circulante e ter o
direito de transmitir, uma esta��o deve esperar o fim do seu temporizador e sentir o
meio f�sico.  Se este estiver livre, ela come�a a emitir imediatamente. Logo depois
desta transmiss�o, o monitor emite um pulso sem espera.

Entre dois pulsos, no m�ximo uma mensagem pode ser transmitida. Para impedir que uma
esta��o de identificador pequeno possa monopolizar o uso do meio indefinidamente,
impedindo as demais esta��es de transmitirem, uma esta��o s� pode emitir novamente
depois que ela observa uma janela livre, isto �, a passagem de dois pulsos
consecutivos sem mensagens intercaladas.

Para garantir que todas as esta��es tenham a oportunidade de transmitir, o per�odo
$T_{sp}$ do pulso na aus�ncia de transmiss�o deve ser igual a $N t_d + 2
T_{prop} + T_s$, onde $T_s$ � um tempo de seguran�a que leva em considera��o o tempo
de transmiss�o do pulso.  ($T_s = 3 \, \mu s$ de acordo com
\cite{Pritty95b}).

A principal vantagem do protocolo TEMPRA � sua simplicidade, o que o torna bastante
confi�vel. Por exemplo, o mecanismo de temporizador permite tolerar a aus�ncia
(falha) de uma esta��o.  No entanto, TEMPRA � um protocolo centralizado que requer a
exist�ncia de um monitor para emitir o pulso.  Portanto, falhas do
monitor podem provocar falhas de comunica��o no barramento.

Uma outra limita��o do TEMPRA � a grande variabilidade dos tempos de rota��o do
bast�o circulante entre o melhor e o pior caso.  Denota-se,
respectivamente $T_{mc}$ e $T_{pc}$ estes dois tempos.  Se houver uma mistura de
mensagens de tamanho m�ximo ($\delta_m = 122,08 \mu s$), t�picas de aplica��es
n�o-cr�ticas, e de mensagens de tamanho $\delta$, t�picas de aplica��es com
requisitos temporais cr�ticos, a variabilidade de $T_{mc}$ e $T_{pc}$ torna-se
significativa.  No melhor caso, o tempo $T_{mc}$ entre duas mensagens consecutivas
da mesma esta��o de identificador $id$ vale: $T_{mc} = \delta + id * t_d + T_s +
T_{sp}$. O tempo $T_{sp}$ corresponde ao tempo necess�rio para que a esta��o observe
uma janela livre.  No pior caso, todas as esta��es (inclusive o monitor) t�m
mensagens de tamanho $\delta_m$ para transmitir. Neste caso, o c�lculo do tempo
$T_{pc}$ envolve a soma dos temporizadores e dos tempos de seguran�a:

\[
T_{pc} = N \, (\delta_m + T_s) + \sum_{id = 1}^{N-1} id * t_d = N \, ( \delta_m + T_s
+ (N-1) t_d/2)
\]
 
Neste c�lculo, os tempos de propaga��o s�o desprezados.

Considerando um exemplo com 10 esta��es num barramento de $500m$ de comprimento numa
rede 100Mbps, os valores para $T_{prop}$, $T_{sp}$, $T_{mc} $ e $T_{pc}$ ficariam tal que:
$T_{prop} \approx 2 \, \mu s$, $T_{sp} \approx 17 \, \mu s$, $26,76 \leqslant
T_{mc} \leqslant 35,76 \, \mu s$ e $T_{pc} \approx 1.295,8 \, \mu s$.
 

\subsection{O protocolo VTPE}
\label{sec:VTPE}

No protocolo VTPE \cite{Carreiro03}, a circula��o do bast�o circulante impl�cito �
organizado atrav�s do uso de contadores locais e de temporizadores. Cada mensagem
que circula no barramento VTPE gera uma interrup��o de \emph{EOF}. Quando observado
por um processo, esta interrup��o provoca o incremento do valor de um contador
local.  Por outro lado, na aus�ncia de transmiss�o, o contador � incrementado depois de
um intervalo de tempo $t_2$ definido em tempo de projeto. O bast�o circulante �
adquirido por uma esta��o quando o valor do seu identificador $id$ � igual ao valor
do seu contador. Portanto, o bast�o circulante � passado tanto explicitamente pela
transmiss�o de mensagem quanto implicitamente pela aus�ncia de transmiss�o. Depois
de um per�odo prolongado sem transmiss�es, um mecanismo baseado em temporizador
garante que a esta��o em posse do bast�o envia uma mensagem vazia. Desta forma,
problemas de sincroniza��o devido aos desvios dos rel�gios locais s�o prevenidos.

Por exemplo, suponha que a esta��o $i$ adquire o bast�o circulante. Os dois casos de
passagens do bast�o circulante seguintes s�o poss�veis:

\begin{description}
\item[Passagem impl�cita] A esta��o $E_i$ n�o tem nada para transmitir.  Neste caso,
  depois de um tempo predefinido $t_2$, as demais esta��es do barramento monitoram o
  meio e constatam que este est� livre.  Portanto, elas incrementam o seus
  contadores locais. Conseq�entemente, a esta��o $E_{i+1}$, sucessora de $E_i$, adquire o
  bast�o circulante.
\item[Passagem expl�cita] A esta��o $E_i$ transmite uma mensagem. A interrup��o gerada
  por esta mensagem provoca o incremento do contador de todas as esta��es do
  barramento, e portanto, o bast�o circulante passa para a esta��o sucessora da esta��o
  $E_i$.
\end{description}

Neste segundo caso, antes de poder emitir, a esta��o que adquire o bast�o deve
esperar um tempo $t_1$ correspondente ao tempo necess�rio para que a mensagem
transmitida pela esta��o $i$ seja processada por todas as esta��es do barramento.

Quando a rede for dedicada exclusivamente � comunica��o de tempo real, as mensagens
t�m um tamanho m�ximo de $\delta$. Adotando as mesmas defini��es de $t_{pc}$ e
$t_{mc}$ usadas na se��o anterior, calculamos que, no melhor caso:

\[
t_{mc} = \delta + t_1 + \sum_{i = 1}^{N-2} t_2 = \delta + T_1 + (N-1) \, t_2
\]

enquanto que, no pior caso, este tempo � dado por:

\[
t_{pc} = \sum_{i = 1}^{ N} (t_1 + \delta) = N \, (t_1 + \delta)
\]

Apesar da variabilidade entre pior e melhor caso, esta abordagem apresenta as
seguintes vantagens:
\begin{itemize}
\item A sobrecarga devido aos temporizadores � menor que a sobrecarga causada pela
  exist�ncia de uma mensagem expl�cita para gerenciar a passagem do bast�o circulante;
\item A flexibilidade do controle do bast�o circulante permite tolerar falhas (n�o h�
  diferen�a entre uma esta��o silenciosa e uma esta��o falha);
\item A estrutura totalmente descentralizada do protocolo n�o apresenta ponto �nico
  de falha.
\end{itemize}

Considerando o mesmo exemplo da se��o anterior e utilizando os valores de $t_2 =
25 \, \mu s$ e $t_1 = 111 \mu s$ de acordo com \cite{Carreiro03}, deduzimos os
seguintes valores: $T_{mc} \approx 341,76 \, \mu s$ e $T_{pc} \approx 1.167,6 \, \mu
s$.

\section{Conclus�o}
\label{sec:sisHibrid}

Neste cap�tulo, diversas solu��es para aumentar a previsibilidade da comunica��o
baseada em Ethernet compartilhada foram apresentadas.  Distinguiram-se notadamente
duas solu��es particulares,TEMPRA e VTPE, pois estas foram as principais fontes de
inspira��o deste trabalho.  Estas duas solu��es permitem resolver o n�o-determinismo
do algoritmo BEB do CSMA/CD. No entanto, a utiliza��o destes dois protocolos para
sistemas h�bridos � limitada pela variabilidade significativa dos tempos de rota��o
do bast�o circulante entre melhor e pior caso.  Entende-se aqui por sistemas
h�bridos, os sistemas compostos de aplica��es com requisitos temporais variados,
cr�ticos e n�o-cr�ticos.

Para completar a an�lise dos sistemas h�bridos, temos que considerar os requisitos
espec�ficos das redes industriais. Neste contexto, alguns dos dispositivos que t�m
requisitos temporais cr�ticos s�o dispositivos eletr�nicos que disp�em de
capacidades de processamento menores que os computadores de uso geral. Tipicamente, a
velocidade do processador destes dispositivos pode ser de duas a tr�s ordens de
grandeza menor que a velocidade dos processadores de computadores. Conseq�entemente,
o tempo de processamento m�ximo de uma mensagem de tempo real de 64 bytes pode ser bem
maior que o tempo necess�rio para a sua transmiss�o. Por exemplo, no protocolo
VTPE~\cite{Carreiro03}, os autores utilizam um tempo de processamento m�ximo de $111
\, \mu s$ para micro-controladores do tipo 8051 com processador de 12MHz.

Para impedir a emiss�o de uma mensagem de tempo real antes do fim do processamento
da �ltima mensagem que foi transmitida, o protocolo VTPE utiliza o temporizador
$t_1$. Este par�metro deve ser maior do que o tempo de processamento do processador
mais lento do barramento.  Depois de uma interrup��o de EOF causada por uma mensagem
cr�tica $m$, a nova esta��o em posse do bast�o circulante espera o tempo $t_1$ antes de
poder transmitir. Desta forma, ela tem certeza que todos os dispositivos conectados
no barramento terminaram de processar a mensagem $m$.

No entanto, durante todo este tempo, o meio f�sico permanece livre.  Isto porque o
tempo de transmiss�o da mensagem cr�tica $m$ (aqui $\delta = 5,12 \, \mu s$) pode
ser muito mais curto que o tempo do seu processamento. Neste caso, o canal de
comunica��o fica dispon�vel para a comunica��o n�o-cr�tica durante o tempo $t_1 -
\delta$.

Esta constata��o motivou o nosso esfor�o para desenvolver um protocolo sem
modifica��o ou adi��o de hardware que possa integrar os servi�os cr�ticos e n�o-cr�ticos num
mesmo barramento Ethernet, otimizando o compartilhamento da banda entre estes dois
servi�os e oferecendo garantias temporais deterministas e constantes para o
servi�o de tempo real.  Esta possibilidade baseia-se na exist�ncia de mem�rias
internas na interface de rede, que permitem a recep��o e a emiss�o ass�ncrona das
mensagens, e, portanto, a libera��o r�pida do meio f�sico.  Esta tecnologia
encontra-se, por exemplo, nos micro-controladores de redes MC9S12NE64, DS80C410 e
DS80C411 \cite{MC9S12NE64, DS80C410}.

� importante observar que solu��es que permitem o controle de acesso ao meio
compartilhado n�o s�o incompat�veis com o uso de comutadores. De fato, o mecanismo
de controle de acesso ao meio permite evitar sobrecargas no comutador, e por
conseguinte, prevenir perdas de mensagens. Apesar de a solu��o proposta neste trabalho
se concentrar no acesso ao meio Ethernet compartilhada, ela � compat�vel com o uso
de comutador, sendo, portanto, adequada a infraestruturas de redes em uso
atualmente.

\begin{comment}

\item Ser implementado por todas as interfaces de rede conectadas no barramento. A
  presen�a de esta��es n�o compat�veis - que executam um outro protocolo - n�o �
  poss�vel.

  De acordo com a classifica��o introduzida por Decotignie \cite{Decotignie05}, este
  ultimo ponto significa que o protocolo \doris{} n�o � compat�vel.  Apesar deste
  fato, um roteador executando o \doris{} e servindo de passarela com o mundo
  Ethernet CSMA/CD pode ser utilizado para permitir a conex�o de um barramento
  \doris{} na Internet.

\end{comment}

% mudar mensagem para quadro IFG -> IPG



%\chapter{O Protocolo \doris{}} % (10 pag)
\label{cap:doris}

%\chapter{A disciplina \emph{DoRiS}} % (10 pag)
\label{cap:specVerif}

%\chapter{Plataforma Operacional}
\label{cap:platOp}

\end{spacing}

%%
%% Parte p�s-textual
%%
\backmatter

% Bibliografia
% � aconselh�vel utilizar o BibTeX a partir de um arquivo, digamos "biblio.bib".
% Para ajuda na cria��o do arquivo .bib e utiliza��o do BibTeX, recorra ao
% BibTeXpress em www.cin.ufpe.br/~paguso/bibtexpress
\renewcommand{\bibname}{REFER�NCIAS}
%\nocite{*}
%\bibliographystyle{authordate2}
%\bibliographystyle{abnt-alf}

\bibliography{../../bib}

%% Fim do documento
\end{document}
