
\cleardoublepage

\thispagestyle{empty}
\begin{center}
  % \sf\large
  \vskip 25mm \vskip\baselineskip
  
  \begin{flushright}
    \begin{minipage}{170mm}
      \begin{tabular}{ r p{0.64\textwidth} }
        &
        {\bfseries\MakeUppercase{TERMO DE APROVA��O}}
        \vspace{2.22cm}\\
        \bf{T�tulo da disserta��o}
        &
        \textsc{Especifica��o formal, Verifica��o e Implementa��o
          de um protocolo de comunica��o determinista, baseado em Ethernet} 
        \vspace{0.74cm}\\
        \bf{Autor}
        &
        \textsc{Paul Denis Etienne Regnier}
        \vspace{0.74cm}\\
        &
        \quotefonti Disserta��o aprovada como requisito parcial para obten��o do grau
        de Mestre em Mecatr�nica, Universidade Federal da Bahia -- UFBA, pela seguinte
        banca examinadora:\\
      \end{tabular}
    \end{minipage}
  \end{flushright}
  \vspace{4cm}

\rule{0.9\textwidth}{0.05em}\\
  \textbf{\textsc{Pr. Doutor George Marconi Lima (Orientador)}}\\ \vspace{0.1cm}
  Ph.D. em Ci�ncias da Computa��o, University of York, Inglaterra\\
  Professor do Departamento de Ci�ncia da Computa��o da UFBA\\ \vspace{2cm}

\rule{0.9\textwidth}{0.05em}\\
  \textbf{\textsc{Pr. Doutor Carlos Montez (Examinador Externo)}}\\ \vspace{0.1cm}
  Doutor em Engenharia El�trica, Universidade Federal de Santa Catarina (UFSC)\\
  Professor do Departamento de Ci�ncia da Computa��o da UFSC\\ \vspace{2cm}

\rule{0.9\textwidth}{0.05em}\\
 \textbf{\textsc{Prof. Dr. Fl�vio Morais de Assis Silva (Examinador PPGM)}}\\ \vspace{0.1cm}
    Dr.-Ing, Technische Universit�t Berlin, Alemanha\\
    Professor do Departamento de Ci�ncia da Computa��o da UFBA\\ 
\end{center}



