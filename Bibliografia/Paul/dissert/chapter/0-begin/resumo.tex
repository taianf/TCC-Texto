\begin{resumo}
  Este trabalho apresenta um protocolo que torna o uso compartilhado de
  Ethernet eficiente para dar suporte aos sistemas de tempo real
  modernos.  O protocolo foi especificado formalmente e sua corre��o
  foi atestada automaticamente atrav�s de um verificador de modelo. Em
  seguida, um prot�tipo foi realizado numa plataforma operacional
  de tempo real.  Os resultados experimentais confirmaram a capacidade
  do protocolo em atender os objetivos definidos na sua proposta.

  As aplica��es que podem se beneficiar deste protocolo s�o
  principalmente aquelas compostas de dispositivos heterog�neos e
  distribu�dos que t�m restri��es temporais de natureza cr�ticas e
  n�o-cr�ticas.  Utilizando o protocolo proposto, tais sistemas podem
  utilizar o mesmo barramento Ethernet de forma eficiente e
  previs�vel. A utiliza��o do barramento � otimizada atrav�s da
  aloca��o apropriada da banda dispon�vel para os dois tipos de
  comunica��o. Al�m disso, o protocolo, compat�vel com os dispositivos
  Ethernet comuns, define um controle descentralizado do acesso ao
  meio que garante flexibilidade e confiabilidade � comunica��o.

  

    \begin{keywords}
     Ethernet, Tempo Real, Toler�ncia a falhas, Especifica��o Formal. 
    \end{keywords}
\end{resumo}

