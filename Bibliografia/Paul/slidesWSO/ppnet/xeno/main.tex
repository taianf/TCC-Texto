\documentclass[msc]{mestrado}

%%\usepackage{fullpage, epic, eepic}
\usepackage{master}

%% Inicio do documento
\begin{document}
\mainmatter

\headsep=40pt
\oddsidemargin=15pt
\begin{spacing}{1.4}
\parskip=6pt

\begin{figure}%
%\begin{sidewaysfigure}
  \centering
  \subfloat[Interrup��o - Sem estresse]{%
    \label{fig:xenSem}%
    {\scalebox{0.6}{\input{fig/xenSem}}}}
  \hspace{7pt}%
  \subfloat[][Escalonamento  - Sem estresse]{%
    \label{fig:xenSemSched}%
    {\scalebox{0.6}{\input{fig/xenSemSched}}}}%
%   \vspace{14pt}%

  \subfloat[Interrup��o - Com estresse UDP e Carga]{%
    \label{fig:xenTot}%
    {\scalebox{0.6}{\input{fig/xenTot}}}}
  \hspace{7pt}%
  \subfloat[][Escalonamento  - Com estresse UDP  e Carga]{%
    \label{fig:xenTotSched}%
    {\scalebox{0.6}{\input{fig/xenTotSched}}}}%

  \caption[Lat�ncia de interrup��o e de escalonamento de Xenomai]{Lat�ncia de
    interrup��o e de escalonamento do Xenomai 2.4-rc5 instalado no Kernel Linux,
    vers�o 2.6.19.7, op��o \ing{Low-Latency}. As figuras \subref{fig:xenSem} e
    \subref{fig:xenSemSched} representam uma execu��o sem estresse. As figuras
    \subref{fig:xenUdp} e \subref{fig:xenUdpSched} representam uma execu��o com
    estresse de comunica��o UDP.  As figuras \subref{fig:xenTot} e
    \subref{fig:xenTotSched} representam uma execu��o com estresse de comunica��o
    UDP e uma carga extra do processador.A freq��ncia de escrita na porta paralela �
    de $20 Hz$.}
  \label{fig:ker2619}%
\end{figure}


\end{spacing}

%% Fim do documento
\end{document}

\begin{figure}[!hbt]
  %\hfill
  %\vspace{0.2in}
  \begin{minipage}[!t]{\textwidth}
    \begin{center}  
      \centerline{\resizebox{0.5\linewidth}{!}{\input{fig/klight1}}}
      \caption{Lat�ncia de interrup��o do \kernell Linux}
      \label{fig:irqKern}
    \end{center}
  \end{minipage}
  %\hfill
  \\
  \vspace{.4in}
  \begin{minipage}[!t]{\textwidth}
    \begin{center}  
      \centerline{\resizebox{0.5\linewidth}{!}{\input{fig/klight2}}}
      \caption{Lat�ncia de interrup��o do \kernell Linux (m�dia sobre 10 valores)}
      \label{fig:irqKernMean}
    \end{center}
  \end{minipage}
  %\hfill
  \\
  \vspace{.4in}
  \begin{minipage}[!t]{\textwidth}
    \begin{center}  
      \centerline{\resizebox{0.5\linewidth}{!}{\input{fig/klight2}}}
      \caption{Lat�ncia de interrup��o do \kernell Linux (m�dia sobre 10 valores)}
      \label{fig:irqKernMean}
    \end{center}
  \end{minipage}
  %\hfill
\end{figure}


% \begin{figure}
%   \begin{center}
%     \input{fig/klight1}
%   \end{center}
% \end{figure}

% \begin{figure}
%   \begin{center}
%     \input{fig/klight2}
%   \end{center}
% \end{figure}

% \begin{figure}
%   \begin{center}
%     \input{fig/klight3}
%   \end{center}
% \end{figure}

