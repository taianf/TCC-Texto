\documentclass[12pt]{article}

\usepackage{sbc-template}

\usepackage{graphicx,url}
\usepackage{color}
\usepackage{epsfig}

\usepackage[latin1]{inputenc}
%\usepackage{amsmath}
\usepackage{amssymb, bm}
\usepackage{verbatim}

%\usepackage{savesym}
%\usepackage{tlatex}
%\savesymbol{implies}
\usepackage{amsmath}
%\restoresymbol{TXF}{implies}

\newcommand{\defeq}{\; \triangleq}

\newcommand{\doris}{\emph{DoRiS}}
\newcommand{\RTS}{R^H}
\newcommand{\SOS}{R^S}

\newcommand{\HW}{\mathcal{W}^H}
\newcommand{\ES}{\mathcal{S}^E}
\newcommand{\RS}{\mathcal{S}^R}
\newcommand{\SW}{\mathcal{W}^S}
\newcommand{\DDP}{\Delta}
\newcommand{\DHW}{\Delta^H}
\newcommand{\DSW}{\Delta^S}
\newcommand{\DES}{\Delta^E}
\newcommand{\DRS}{\Delta^R}

%\sloppy

\title{Deterministic Integration of Hard and Soft Real-Time Communication
       over Shared-Ethernet}

\author{Paul Regnier \inst{1}
  \hspace{0.2cm} George Lima \inst{1}
  \email{\{pregnier, gmlima\}@ufba.br}
  %\thanks{IARA - Computer Science and Mechanical Engineering Departments}
}

\address{
  Distributed Systems Laboratory (LaSiD) - Computer Science Department (DCC)\\
  Pos-Graduation Program on Mechatronics - 
  Federal University of Bahia\\
  Campus de Ondina,
  40170-110, Salvador-BA, Brazil
}

\begin{document}

\maketitle

\begin{abstract}
    This paper presents a protocol proposal that makes Ethernet
    suitable for supporting modern real-time systems. Applications
    that could benefit from the proposed protocol are mainly
    those composed of heterogeneous distributed components, which
    are specified in terms of both hard and soft timing constraints.
    Indeed, using the proposed protocol, hard and soft real-time
    tasks can efficiently and predictably share an Ethernet bus.
    The bus utilization can be optimized by an appropriate
    allocation of the available bandwidth into hard and soft
    communication. Moreover, the protocol, compatible with standard
    Ethernet hardware, can implement a decentralized medium
    access control, increasing communication flexibility and
    reliability.
\end{abstract}

\section{Introduction}

Research on real-time systems have been facing several challenges
in the last few decades. From the application domain perspective,
such systems, once seen as composed of simple, strict and short
hard periodic tasks, typically used for control
loops, nowadays include applications such as telecommunication,
industry, military, aerospace, assets monitoring, automobile etc.
While parts of these systems may contain the usual hard periodic
tasks, requiring known bounded response time and controlled
jitter, other parts may be made of much more complex soft (or
non-hard) tasks to deal with sporadic or aperiodic events
\cite{Liu00}, leading to a considerable task diversity.
Further, from the point of view of their structure,
real-time systems have been moving from centralized to distributed.
This requires a suitable communication network capable of dealing
with the new demands task diversity requires. Indeed, network
protocols must deal with different traffic patterns and must provide
not only controlled jitter and bounded message transmission time,
as required by the usual hard tasks, but also high throughput,
as demanded by soft and other non-periodic tasks. As a result,
industrial communication networks, known by being reliable and
predictable, may not suffice anymore due to their usual low bandwidth.

Ethernet has been considered a promising alternative for
designing distributed real-time systems. It provides high
bandwidth, it is cheap, efficient and widely used. However, one
cannot consider its direct use in the real-time systems domain.
This is due to its {\em probabilistic} bus arbitration scheme,
used by the Carrier Sense Multiple Access with Collision Detection
protocol (CSMA/CD) ~\cite{CSMA/CD01, Dolejs04}. Several solutions
to this lack of determinism, either based on hardware or software
modification, have been proposed along the last three decades. A
brief description on related work is given in section
\ref{sec:relatedWork}, although interested readers can refer to
\cite{Hanssen03, Decotignie05} for details. To the best of
our knowledge, none of those former approaches succeed to offer
hard real-time guarantees, high throughput and fault tolerance
mechanism using a decentralized medium access control
without hardware modification.

This paper describes a broadcast-oriented software-based 
shared-Ethernet protocol built on top of the Ethernet layer.
The protocol, presented in section
\ref{sec:doris}, provides deterministic communication guarantees
without jeopardizing flexibility neither increasing hardware costs.
Also, it preserves the Ethernet broadcast communication capacity,
facility required by many real-time systems. The protocol is
reliable, predictable, efficient, optimizes the bus utilization,
and is capable of dealing with different communication patterns
(eg periodic and aperiodic). By providing an innovative dynamic
reservation mechanism, the protocol also offers an adaptable
bandwidth allocation to hard real-time traffic.


\section{Software-based Ethernet Approaches to Real-Time Ethernet
         \label{sec:relatedWork}}

Perhaps the simplest deterministic software-based solution is to
divide time into a sequence of slots, each of which is assigned to
a specific system node. Nodes can transmit their message only in
their respective time slot. This scheme, known as Time Division
Multiple Access (TDMA) \cite{Kurose05}, gives a great deal of
communication predictability. This characteristic suits well
industrial applications, composed mainly of hard periodic tasks.
However, TDMA lacks flexibility since even if a station
has nothing to transmit, its time slot will be wasted.

Token-based protocols are usually more flexible than TDMA. They
use a token to give permission to a node to transmit messages. The
nodes are organized in a logical ring and the token
rotates on the ring. Only the token holder node has the right to
transmit on the bus. The token can be explicit or implicit.
Explicit token uses a token message to allow nodes to transmit.
Whenever a node terminates a transmission, it sends the token to
the next node according to the schedule. RETHER
\cite{Venkatrami94} and RTnet \cite{Hanssen05} are examples of
explicit-token Ethernet-based protocols. Although these
protocols deal with different traffic patterns, they introduce
extra overhead in case of token loss.
Implicit-token protocols based on Timed Packet Release mechanism
(TPR) offer a better alternative \cite{Pritty95, Carreiro03,
ControlNet97}. In brief, a token is implicitly passed on to the
next node independently of whether or not the previous node on
the ring transmits its message. The absence of
message means that a node is giving up its right to use the bus.
Thus, the bus can be utilized more efficiently.

Other protocols, like FTT \cite{Pedreiras02}, use a master-slave
model to control bus access. Here a special node plays the role of
an arbiter (the master), polling other nodes (the slaves). Slaves
are allowed to transmit their messages only if they get permission
to do so. This model can handle hard and soft traffics
but exhibits single point of failure, which can be dealt with
by master replication. In any case, there is an asymmetric load
distribution on the net, which may imply problems of scalability
and performance.

Another way to avoid collisions is
switch-based solutions \cite{Kopetz05}. However, the use 
of switches introduces routing, forwarding and
buffering delays. Also, broadcast communication implementation
is not as easy as in shared-Ethernet and require special
care to manage traffic congestion in the switch buffers
\cite{Wang02, LoBello01}. 

In this paper we mix two approaches together, TDMA and implicit
token, to get the best out of each one. TDMA is used to create
two windows to predictably deal with hard and soft
messages, respectively. Inspired by the Virtual Token Protocol
(VTPE) \cite{Carreiro03}, the bus access in each window is
controlled by organizing the nodes in two logical rings, one for each
window, and by using implicit token so that bus utilization can be
optimized. The proposed protocol, called \doris{}, which stands
for {\em Double Ring Service for Real-Time Ethernet}, is flexible
and efficient.


\section{DoRiS}
\label{sec:doris}

\doris{} is a collision-free protocol built on top of Ethernet
hardware. Located between the IP and the physical layers of the
Internet stack, the protocol works as a logical layer, extending
the CSMA/CD MAC layer.
\doris{} was designed to support hybrid systems where industrials
sensors, actuators and controllers share the communication network
with other soft applications. It is
important to notice that the processing speed and the
communication characteristics of these two types of application
may differ to a great extent. Indeed, most industrial
appliances available on the shelf have low processing capacity
compared to the Ethernet bandwidth. For example, Carreiro et al.
[2003] consider typical 8051 microcontrollers that spend up to
$111 \, \mu s$ to process an incoming 64B message. As the
transmission time on a 100Mbps Ethernet link of such
message is $5.12\,\mu s$, such an appliance allows only about
$4.6\%$ of bus utilization for hard real-time communication.
On the other hand, multimedia application requires higher
transmission rates and faster processing nodes. Considering these
observations, \doris{} provides both temporal
guarantees for hard real-time tasks and high bandwidth access
to soft and best effort ones.

After the computation model, the \doris{} scheme will
be presented in details in the next sections. We will
focus the protocol description on the functions that guarantee
the shared medium access without collision. As they are 
based on the Ethernet standard, the functions to receive
messages will not be described. Important mechanisms such as
node synchronization, group membership and fault tolerance will
be adressed in future work.


\subsection{Computation Model
            \label{sec:model}}

We consider a set of nodes connected through a shared-Ethernet bus.
Each node is uniquely identified in the system and may run  
hard real-time or soft tasks. For the sake of notation, we refer to
the former simply as {\em tasks} while to the latter as {\em
processes}. Let $N_H$ ($N_S$) be the numbers of hard tasks
(soft processes). We define the two following ordered sets
$\RTS = \{T_1,T_2,\ldots,T_{N_H}\}$ and
$\SOS =\{P_1,P_2,\ldots,P_{N_S}\}$.
These sets are the two logical rings of \doris,
where a single token rotates.

Received messages are processed by the tasks within a maximum
processing time, denoted $\pi$. This time is usually associated to
message processing by industrial appliances (micro-controllers,
sensors etc). We assume that these appliances, called hereafter
{\em slow nodes}, need temporal guarantees and that their messages
are short, usually periodic, and have hard real-time constraints.
Such messages, called \emph{hard messages}, are assumed to have a
constant length of 64B and are transmitted through the network
within a maximum transmission time $\delta \ll \pi$.
The processing time for message transmission is
assumed to be included in the task computation time and so 
is ignored during the protocol description.

Our protocol uses the pu\-blish-sub\-scribe com\-muni\-ca\-tion
model accor\-ding to which when\-ever a task or
process has a message to send, it transmits such a message using the
Ethernet broadcast standard address (48 address bits set to 1)
\cite{Dolejs04}.
When various applications are located at the same node,
they uniquely identify their messages using the
type field of the Ethernet frame.
Each task/process receives and processes every message.
Based on the source address, they then decide whether they are
interested in the message. In practical applications, tasks may
not have to process all messages or may drop unused ones.
However, for the sake of model simplification, we assume here that
all hard real-time messages are fully processed by the tasks.

As mentioned before, if only slow nodes are present in the
\doris{} segment, the bus may be under-utilized. However, if there
are fast nodes connected to the bus, \doris{} allows them to use
this spare bandwidth. Assuming that nodes are equipped with
receiving buffers, receiving and processing hard messages are
independent actions that can happen simultaneously. This also
implies that more than one message can be sent in a row.
Therefore, the maximum bandwidth rate available for hard real-time
traffic is $B_m = \delta / \pi$. It is important to notice that
$B_m$ indicates the bound above which buffer overflows occur since
the slowest node will receive more hard messages than it
can process.

\subsection{The Protocol Structure}

The communication on the \doris{} bus is structured as a
succession of time intervals, called \doris{} {\em segment}
and denoted $DS_k$ for $k \in \mathbb{N}^*$.
Each $DS_k$ is itself divided into two windows,
$\HW_k$ and $\SW_k$, respectively associated to hard and
soft real-time traffics. Tasks send messages in $\HW$ and
processes use $\SW$ to transmit theirs. The size of  $\HW$ and $\SW$
are denoted $\DHW$ and $\DSW$ and the
\doris{} segment size is defined by $\DDP = \DHW + \DSW$.

We define a \doris{} \emph{rotating sequence} $RS_k$ associated
to $DS_k$ as the ordered list of $N_H$ consecutive \doris{} segment
that begins by $DS_k$, ie $RS_k = (DS_k, \dotsc , DS_{k+N_H-1})$
as illustrated in Figure \ref{fig:dorisStruct} focusing on $DS_i$.

\begin{figure}[ht]
  \index{fig!dorisStruct}
   \centering
  \input{fig/dorisStruct.pstex_t}
  \caption{The $\mathbf{N_H}$ \doris{} \emph{segments} of
           $\mathbf{RS_1}$ with a focus on $\mathbf{DS_i}$
  \label{fig:dorisStruct}}
\end{figure}

We assume that tasks and processes are equipped with local timers,
respectively denoted $t_{T_i}$ and $t_{P_i}$, instances of the node timer.
For simplicity, if $T_i$ (or $P_i$) and $T_j$ (or $P_j$) are two tasks
(processes) located at the same node, their local timers,
instances of the node timer, are equal.
To divide the bandwidth into \doris{} segments, these timers assume
values from $0$ to $\DDP$ and are reset periodically at the end
of each $DS_k$.
Each task $T_i$ and process $P_i$ maintains also local counters,
respectively denoted $K_{T_i}$ ($1 \leqslant K_{T_i} \leqslant N_H$)
and $K_{P_i}$ ($1 \leqslant K_{P_i} \leqslant N_S$).
Local timers and counters are used to organize the token 
circulation between $\RTS$ and $\SOS$ as will be seen in
the next two sections.

To allow for some flexibility and scheduling policy, the hard
real-time window $\HW_k$ is further divided into two slots, the
elementary ($\ES_k$) and the reservation ($\RS_k$) slots, as
shown in Figure \ref{fig:dorisStruct} as for $DS_i$.
Messages sent in these two slots are hard messages, respectively
called elementary and reservation messages. $\RS$
is used to offer a reservation mechanism, which will be explained
in section \ref{sec:reservation}.

\subsection{The Hard Real-Time Ring}

\subsubsection{The Elementary Slot}

Tasks are logically organized as a ring where a unique implicit
token rotates, granting medium access authorization according
to temporal and logical rules.

Consider a slot $\ES_k$. We define the mapping $Id$ from
$\mathbb{N}^*$ to $\{1, \ldots , N_H\}$ as $Id(k) = (k-1)\, mod(N_H)+1$.
Let $i = Id(k)$, $T_i$ is said to hold the token in $\ES_k$
if its counter $K_{T_i}$ equals $i$ and if its timer is equal to zero
(ie when $\ES_k$ just begins). For instance, considering $N_H = 4$,
$T_3$ holds the token during $\ES_3$, $\ES_7$, $\ES_{11}$ and so on.
The following predicate formalizes these conditions:

\begin{tabular}{ l l }
  $ElemHolder(i) \defeq$ & $\; (t_{T_i} = 0) \quad
                             \wedge \quad (K_{T_i} = i)$
\end{tabular}

Whenever $ElemHolder(i)$ becomes true, $T_i$ broadcasts an
elementary message by executing the $sendElemMsg(i)$ operation,
which consists of transmitting its local counter $K_{T_i}$, its
reservation list (see next section), and its optional data if
there are any. Sending an elementary message in each $DS_k$ makes it
easier to implement failure detectors and gives communication
regularity, characteristics required by hard real-time tasks.
Clearly, the logical condition $t_{T_i} = 0$ does not consider 
processing delays and should be expressed by means of time
interval for implementation.

The protocol ensures the following invariant.
In each $\ES_k$, for all $T_i$, the counter $K_{T_i}$
equals $Id(k)$. This is achieved by means of (periodically)
incrementing the local counters by 1 at the end of each $DS_k$.

Observe that, as each $DS_k$ contains one elementary message, tasks
can synchronize themselves by using the End-Of-Frame interrupt of
the broadcast message. Also, if an elementary message suffers an
omission fault, the progression of the local counters will not be
blocked, as they rely on local timers. As a consequence, the
maximum drift of local timers has to be bounded according to the
synchronization protocol and the number of consecutive omissions
assumed to be tolerated. We are currently dealing with these aspects
of the protocol and will not adress them in this paper.

\subsubsection{The Reservation Slot}
\label{sec:reservation}

The reservation scheme is used to implement some {\em classes}
of message by means of the reservation slots, which make
some extra bandwidth available for the tasks.
To implement this scheme each task $T_i$ sends its
reservation list upon the execution of $sendElemMsg(i)$ 
(cf. previous section).
This list is a possibly empty subset of the ordered set
$\{i+1,\ldots, N_H, 1,\ldots, i-1\}$ that indicates the reservation
slots in $RS_k$ where $T_i$ will transmit extra data.
For example, with $N_H = 4$, reservation list $\{3,1\}$ sent
by $T_2$ in $\ES_6$ means that $T_2$ requests $\RS_7$ and
$\RS_9$ in $RS_6$. If $T_i$ sends an empty list,
no reservation is requested.

When a task $T_i$ receives a reservation list sent in $DS_k$,
it updates a boolean vector of size $N_H$, denoted $\Gamma_i$.
If the entry $\Gamma_i[j=Id(k')]$ is true, $\RS_{k'}$ is reserved.
Otherwise, $\RS_{k'}$ is free. The value true for $\Gamma_i[j]$
is only valid for all \doris{} segments in $RS_k$. This implies
that $\Gamma_i[j]$ is reset after the end of $DS_{k+N_H-1}$. 
It is important to note that {\em at least one reservation slot
per $RS_k$ ($k = 1,2,\ldots$) is guaranteed per task}.
Indeed, $T_i$ ($i =Id(k)$) is the first task which can
request the reservation of $\RS_{k+N_H-1}$.

For a task to transmit its message in a reservation slot,
it has to be sure that the reservation procedure was successful
and such a reservation slot is not reserved by any other task.
For example, omission faults of elementary messages could let
$\Gamma_i$ be inconsistent.
The protocol deals with these requirements in an efficient way
by establishing a reservation condition.
Indeed, $T_i$ may only request a reservation in $\ES_k$ if its
{\em state is consistent}, which is formalized by the following
predicate for $k > N_H$:

\begin{tabular}{ l p{9cm} }
 $Consistency(i,k) \defeq$ & $T_i$ \emph{received all elementary
       messages sent in} $RS_{k - N_H}$\\
\end{tabular}

For $k \leqslant N_H$, $T_i$ is consistent if it received all 
elementary messages since $DS_1$.

Let $\DES=\delta + IPG$ be the size of $\ES$, where IPG is the Inter
Packet Gap \cite{CSMA/CD01}. A task $T_i$ holds the token
in $\RS_k$ if it has a reservation for $\RS_k$ and its timer $t_{T_i}$
equals $\DES$. Formally,

\begin{tabular}{ l p{8cm} }
  $ReservHolder(i, k) \defeq$ & $(t_{T_i} = \DES)$
                $\;\; \wedge \quad (\Gamma_i[K_{T_i} = Id(k)] = true)$\\
\end{tabular}

\begin{comment}
As can be noted, the reservation list size is up to $N_T-1$ and can
be implemented as a bit vector. An efficient alternative
for (implicitly) implementing this list
is to send the number of windows requested.
In this case, a request carrying $k > 0$ would mean that the next $k$
free windows were to be reserved. This would imply that
$\lceil \log_2 (N_T-1) \rceil$ message bits should be used to
represent the implicit list instead of $N_T-1$ bits.
Nevertheless, one can find out
a trade-off between the list size and the maximum number of $\RS$ per
task that could be reserved. We consider here the list size as an
implementation detail and do not impose any restriction to it.
\end{comment}

Two interesting properties must be highlighted.
First, the token rotation time is constant and equals $N_H \DDP$,
which gives communication regularity for the tasks.
Second, there are three implicit message
classes: one that uses elementary slots only;
the other is regarding the guaranteed reservation slot;
and the third that makes use of the free (but not guaranteed)
reservation slots. These message classes makes the protocol
more flexible when compared to the TDMA approach.  

\subsection{The Soft Ring $\SOS$}

As in the hard ring, processes use their timers to control
the soft ring. Elementary messages, as they are periodic and of
constant size, are used as a temporal reference by all processes.
In brief, the EOF of every elementary message is used as a \emph{pulse}
which define the beginning instant of the next $DS_k$.
Like the VTPE protocol \cite{Carreiro03}, this is actually a
decentralized version of the Timed Packet Release (TPR) mechanism
\cite{Pritty95} since no node/process plays a central role for sending
the pulse.

Unlike the hard ring, processes do not have to always send messages.
We introduce $d_r$, a temporal parameter of the protocol.
The token rotates based on the carrier sense mechanism provided by
Ethernet, using the following TPR mechanism.
From the beginning of each $\SW$, 
for each period $d_r$ that it senses the medium idle, a process
increments its local counter $K_{P_i}$.
Otherwise, some other process is transmitting and the medium is busy.
In this case, each process waits for the
EOF interrupt associated to the current transmission and then
increments its counter.
The parameter $d_r$ is chosen to be $1 \, \mu s$ (considering a 100Mbps bus).
This ensures unambiguous detection of the start of a packet
transmitted by a previous node \cite{Pritty95}.
Hence, a process $P_i$ is allowed to transmit when the following
spredicate holds:

\begin{tabular}{ l p{9.7cm} }
  $SoftHolder(i) \defeq$ & ($\DHW \leqslant t_{P_i} < \DHW + \DSW) \;
       \wedge \;$ Medium is idle $\; \wedge \;(K_{P_i} = i)$\\
\end{tabular}

While the token rotates, the remaining time
in $\SW$ decreases. Thus, a process $P_i$ could receive the token
without enough time to transmit its message in the current $\SW$.
As this could happen an arbitrary number of times, a process might
indefinitely be unable to transmit its message.
To solve this problem, we define a ``STOP'' message as a special
soft message (64B) that stops the token time passing in the current $\SW$.
Hence, a STOP message is used to distinguish a process
that has nothing to send from a process that has not enough time
to transmit. If $P_i$ sends a STOP message, it will be the
first to hold the token in the next $\SW$.
As $\SW$ ends when the remaining time is equal to $\delta$,
a process that receives the token always has enough time in the
current $\SW$ to send a STOP message.

Considering that a process $P_i$ may suffer a crash and then
recovers, its local counter $K_{P_i}$ may be out of date when $P_i$
recovers.
There are two cases that have to be
dealt with. First, when an empty $\SW$ happen, the process with
the smaller identifier is assumed to hold
the token first in the next $\SW$. This is done by
resetting $K_{T_i}$ to 1 at all processes. Second,
if $P_i$ observes a soft message transmission, it
identifies the message sender from the source address 
and type field of the message.
Using the $DS_k$ beginning instant and the message Start-Of-Frame
instant, $P_i$ localizes the token.
As can be noted, the protocol is safe in both cases.

In the worst case, if only one process wants to transmit, it can suffer the
maximum delay of $N_H \, d$. If all processes wants to transmit,
the last to receive the token may wait the worst token rotation time
of $(N_H - 1) \DSW$.

As for the bandwidth allocation, it is important to choose
adequate values for the window sizes. As two hard messages may
be sent in each $DP$, each slow node needs $2 \pi$ time units to
process them. Thus, as there are buffers,
$\DSW \geqslant 2 \pi - \DHW$ must hold.
The bandwidth allocation for hard messages is maximized if
one chooses $\DSW = 2 \pi - \DHW$. This gives
$B_{h} = 2 \delta /
\DDP$ of the bandwidth for the hard ring.
As $\DDP = \DHW + \DSW$, $B_{h} = \delta / \pi$. Assuming $\pi = 111\mu s$,
as in \cite{Carreiro03}, $B_{h} = 4.6\%$.
The remaining bandwidth (95,4\%) is available for the soft ring.

\section{Conclusion}
\label{sec:conclusion}

This paper has described \doris{}, a protocol based on two logical rings
that allows the co-existence of both hard and
soft real-time traffics on the same Ethernet bus.
Both types of traffic are isolated from each other in the time domain
and the bus utilization can be optimized.
By using the TDMA approach to implement the hard ring, \doris{}
achieves predictability for hard real-time applications.
The implicit token approach was used to control the soft ring.
This configuration can allow for a better bandwidth
allocation. Using typical parameters for slow nodes, it has been shown that
\doris{} can allocate up to $95\%$ of the bandwidth for soft traffic.

The described protocol is currently being formally specified and
then will be implemented, possibly using some open source real-time
operating system. Further protocol functionalities must be considered
in future work. For example, appropriate priority policies can be
used to take advantage of the reservation mechanism. Also, mechanisms
such as membership control and dynamic reconfiguration can increase
significantly the reliability and flexibility of
\doris{}. They will be present in future steps of our research.

\textbf{Acknowledgment}.
We would like to thank Fl�vio Morais de Assis Silva,
Francisco Borges Carreiro and the anonymous
referees for their valuable advice and comments. We
also acknowledge the Brazilian funding agencies CNPq (grant
304084/2003-4) and FAPESB (grant 3381/2005) for their financial support.

%\small
\bibliographystyle{sbc}
\bibliography{bib}

\end{document}

In order to allow any size of message inside the range of the
Ethernet standard (from 64 to 1518 bytes), which corresponds to a
temporal range from $\delta = 5.12$ to $\delta_m = 121.44 \, \mu
s$, $2 \pi$ has to be greater than $\delta_m$.
This restriction is satisfied by the values $\pi = 111 \; \mu s$
assumed by \cite{Carreiro03}.
