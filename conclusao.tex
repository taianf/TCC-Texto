\xchapter{Conclusão}{}\label{cap5}

Este trabalho apresentou os tipos de mecanismos de interrupção presentes no Linux e também algumas estratégias de adaptação do kernel do Linux para se tornar um sistema operacional de tempo real. Analisamos as diferenças de latência em um kernel de tempo real e em um kernel Linux comum para Raspberry Pi. Também analisamos como a carga do sistema influencia essas latências.

A plataforma analisada, o Raspberry Pi 3 Model B, é um sistema compacto e com um ótimo custo-benefício. Sua flexibilidade a torna a plataforma de computação preferida em muitos projetos de sistemas embarcados personalizados. Nos sistemas em que o determinismo temporal dos cálculos é necessário, é possível utilizá-lo como plataforma, desde que seja bem escolhida a maneira pela qual as interrupções serão tratadas e que tipo de kernel deverá ser utilizado.

\section{Dificuldades} \label{Dificuldades}

Durante a concepção deste trabalho, o escopo inicial consistia em comparar o determinismo entre o patch Preempt-RT e o Xenomai, ambos patches em tempo real para Linux, mas com abordagens diferentes. Por ser o patch oficial e disponível no repositório Raspbian, o Preempt-RT foi facilmente compilado e teve pouca dificuldade em adaptar o INTSight.

O Xenomai, no entanto, exigiu um esforço adicional para compilar. Somente encontrando o trabalho de Johansson \cite{Johansson2018}, que passou pelas mesmas dificuldades, eu pude compilar Xenomai para Raspberry. No entanto, ainda era necessário adaptar o INTSight à estrutura de nanokernel, pois o acesso aos pinos GPIO é diferente nessa arquitetura. Como não foi possível adaptar a tempo para este trabalho, as medições de Xenomai tiveram que ser removidas do escopo deste trabalho.

\section{Trabalhos futuros}

O escopo inicial da inclusão do Xenomai não pôde ser alcançado neste trabalho, mas espera-se que, em um futuro próximo, seja possível retornar ao trabalho e concluir a adaptação do INTSight a ele. Embora não seja a versão oficial, existem kernels pré-compilados para o Raspberry Pi com o Xenomai. Em trabalhos anteriores com versões mais antigas do Linux e com um sistema x86, como feito por \cite{Regnier2008}, mostram que o Xenomai tinha um determinismo melhor que o Preempt-RT, por isso é interessante que essa análise seja feita em um kernel mais recente e em uma plataforma diferente.

\section{Trabalhos relacionados}

Em 2003, Matthew Wilcox \cite{Wilcox2003} apresentou estruturas de interrupção e como elas fazem para enfileirar tarefas de interrupção que precisam ser tratadas fora da zona de tempo crítico e como essas estruturas gerenciam os dados.

Em 2008, no Workshop de Sistemas Operacionais, Paul Regnier et al \cite{Regnier2008} apresentaram uma avaliação das interrupções no Linux. Eles compararam a latência de ativação no kernel padrão, Preempt-RT e Xenomai. Eles avaliaram os tempos de latência da interrupção, o tempo até a execução do prólogo e a latência de ativação, o tempo até a execução do epílogo, usando 3 estações em rede para realizar as medições, sendo uma estação de medição, uma estação para gerar eventos de interrupção na estação de medição e outra para gerar uma carga na estação de medição. Eles chegaram à conclusão de que o Xenomai era muito mais estável quando a estação de medição estava sob uma carga de trabalho.

Em 2016, Tomaž Šolc comparou o tempo de resposta do Raspberry com o tempo de resposta do Arduino \cite{Solc2016}. Ele realizou testes usando um Arduino Uno e um Raspberry Pi Zero. O Arduino foi testado tanto para interrupções de hardware quanto em polling. Os testes no Raspberry foram feitos com um módulo do kernel, um programa C nativo e outro programa Python. O Arduino, sendo um microcontrolador, alcançou tempos de resposta quase constantes quando usado com interrupções e uma pequena variação com polling. O Raspberry teve tempos de resposta comparável quando no modo kernel, mas com valores oscilantes, tornando-o menos previsível.

Em 2018, Gustav Johansson \cite{Johansson2018} fez uma avaliação da latência do Raspberry usando o Xenomai. Ele fez várias análises, mas testou apenas o Xenomai, sem comparar com outros sistemas. Ele usou um kernel pré-compilado para realizar as medições, pois teve vários problemas para aplicar o patch e compilar o kernel. Inicialmente, vários problemas foram encontrados na compilação do kernel com o Xenomai durante o curso deste trabalho, e o guia que Gustav forneceu no apêndice do seu trabalho foi essencial para ter sucesso na minha compilação.

Também em 2018, Luis Gerhorst \cite{Gerhorst2018} construiu uma ferramenta para executar análise de latência de interrupção no Linux, que é dividida nos módulos INTSpect e INTSight. Esta ferramenta desenvolvida por ele foi usada para realizar as medições deste trabalho. Em seu trabalho, ele mostrou a viabilidade da ferramenta testando em uma placa \textit{SAMA5D3 Xplained}. Esta placa possui um processador ARM de núcleo único, enquanto o Raspberry possui quatro núcleos em sua CPU.

